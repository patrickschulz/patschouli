\newif\ifcircuitsflip

% drawing styles
\tikzset{
    wirethickness/.style = {thick},
    wire/.style={wirethickness},
    arrowwire/.style = {
        wirethickness,
        ->,
        >=stealth,
    },
    lwire/.style={wirethickness, line cap = rect},
    scriptsize/.style={font=\scriptsize},
    ---/.style args={#1 and #2}{to path={-- ++(#1, 0) |- ($(\tikztotarget)+(#2, 0)$)}},
    ---/.default = 0cm and 0cm,
    -|-/.style args={#1 and #2}{to path={-| ([xshift=#1, yshift=#2]$(\tikztostart)!0.5!(\tikztotarget)$) |- (\tikztotarget) \tikztonodes}},
    -|-/.default = 0cm and 0cm,
    |-|/.style args={#1 and #2}{to path={|- ([xshift=#1, yshift=#2]$(\tikztostart)!0.5!(\tikztotarget)$) -| (\tikztotarget) \tikztonodes}},
    |-|/.default = 0cm and 0cm,
    angleconnect/.style={to path={-- ++(#1, 0) -- ($(\tikztotarget)-(#1, 0)$) -- (\tikztotarget)}},
    angleconnect/.value required = true,
    tmodule/.style args ={#1 and #2}{draw, rectangle, wire, minimum width = #1, minimum height = #2},
    tmodule/.default = {0cm and 0cm},
    module/.style args ={#1 and #2}{
        wire,
        draw, rectangle,
        align = center,
        minimum width = #1,
        minimum height = #2
    },
    module/.default = 1cm and 1cm,
    sumpoint/.style = {
        wire,
        draw, circle,
        minimum size = 0.5cm,
        inner sep = 0cm
    },
    current arrow/.style = {
        postaction = decorate,
        decoration = {
            markings,
            mark = at position #1 with {\arrow{stealth}}
        }
    },
    current arrow/.default = 0.5,
    current arrow reversed/.style = {
        postaction = decorate,
        decoration = {
            markings,
            mark = at position #1 with {\arrowreversed{stealth}}
        }
    },
    current arrow reversed/.default = 0.5,
    % flip components (effect depends on shape)
    flip/.is if = circuitsflip
}

\pgfkeys{
    % general circuits options
    /tikz/circuits/line width/.initial = 0.8pt,
    /tikz/circuits/dot/radius/.initial = 0.05cm,
    /tikz/dot/.style = {dotshape},
    /tikz/circuits/port/radius/.initial = 0.05cm,
    %/tikz/port/.style = {portshape}
    /tikz/port/.style = {wire, draw=black, fill=white, circle, inner sep = 0cm, minimum size = 0.1cm}
}

\makeatletter
\pgfdeclareshape{dotshape}
{
    \saveddimen{\radius}{\pgf@x=\pgfkeysvalueof{/tikz/circuits/dot/radius}}
    \anchor{center}{\pgfpointorigin}
    \anchor{north}{\pgfpointorigin}
    \anchor{south}{\pgfpointorigin}
    \anchor{west}{\pgfpointorigin}
    \anchor{east}{\pgfpointorigin}
    \backgroundpath{
        \pgfpathcircle{\pgfpointorigin}{\radius}
        \pgfusepath{fill}
    }
}
\pgfdeclareshape{portshape}
{
    \saveddimen{\radius}{\pgf@x=\pgfkeysvalueof{/tikz/circuits/port/radius}}
    \anchor{center}{\pgfpointorigin}
    \backgroundpath{
        \color{white}
        \pgfsetstrokecolor{black}
        \pgfsetlinewidth{\pgfkeysvalueof{/tikz/circuits/line width}}
        \pgfpathcircle{\pgfpointorigin}{\radius}
        \pgfusepath{fill, stroke}
    }
}
\makeatother

\input{symbols/ota}
\input{symbols/opamp}
\tikzset{
    circuits/capacitor/width/.initial = 0.1cm,
    circuits/capacitor/height/.initial = 0.5cm,
    capacitor/.style = {capacitorshape, anchor = plus}
}

\makeatletter
\pgfdeclareshape{capacitorshape}
{
    \saveddimen{\width}{\pgf@x=\pgfkeysvalueof{/tikz/circuits/capacitor/width}}
    \saveddimen{\height}{\pgf@x=\pgfkeysvalueof{/tikz/circuits/capacitor/height}}
    \savedanchor{\plus}{
        \pgfpointadd{\pgfpointorigin}{\pgfpoint{-0.5 * \pgfkeysvalueof{/tikz/circuits/capacitor/width}}{0cm}}
    }
    \savedanchor{\minus}{
        \pgfpointadd{\pgfpointorigin}{\pgfpoint{0.5 * \pgfkeysvalueof{/tikz/circuits/capacitor/width}}{0cm}}
    }

    % electrical terminals
    \anchor{plus}{\plus}
    \anchor{minus}{\minus}
    % general anchors
    \anchor{center}{\pgfpointorigin}
    \anchor{east}{\minus}
    \anchor{west}{\plus}
    \anchor{north}{\pgfpointadd{\pgfpointorigin}{\pgfpoint{0cm}{0.5 * \height}}}
    \anchor{south}{\pgfpointadd{\pgfpointorigin}{\pgfpoint{0cm}{-0.5 * \height}}}
    \anchor{south east}{\pgfpointadd{\pgfpointorigin}{\pgfpoint{0.5 * \height}{-0.5 * \width}}}
    \anchor{south west}{\pgfpointadd{\pgfpointorigin}{\pgfpoint{-0.5 * \height}{-0.5 * \width}}}
    \anchor{north east}{\pgfpointadd{\pgfpointorigin}{\pgfpoint{0.5 * \height}{0.5 * \width}}}
    \anchor{north west}{\pgfpointadd{\pgfpointorigin}{\pgfpoint{-0.5 * \height}{0.5 * \width}}}
    \beforebackgroundpath{
        \pgfsetlinewidth{\pgfkeysvalueof{/tikz/circuits/line width}}
        \pgfpathmoveto{\pgfpointadd{\plus}{\pgfpoint{0cm}{0.5 * \height}}}
        \pgfpathlineto{\pgfpointadd{\plus}{\pgfpoint{0cm}{-0.5 * \height}}}
        %\pgfusepath{stroke}
        \pgfpathmoveto{\pgfpointadd{\minus}{\pgfpoint{0cm}{0.5 * \height}}}
        \pgfpathlineto{\pgfpointadd{\minus}{\pgfpoint{0cm}{-0.5 * \height}}}
        \pgfusepath{stroke}
    }
}
\makeatother

% vim: ft=plaintex

\input{symbols/resistor}
\input{symbols/inductor}
\newif\ifmosfetdrawbulk
\newif\ifmosfetdrawbulkarrow
\newif\ifmosfetmirror
\newif\ifmosfetnoarrow
\newif\ifmosfetfilldot

\tikzset{
    circuits/mosfet/scale/.initial=1, % not implemented
    circuits/mosfet/gate width/.initial={0.6cm},
    circuits/mosfet/gate skip/.initial={0.1cm},
    circuits/mosfet/drain length/.initial={0.2cm},
    circuits/mosfet/source length/.initial={0.2cm},
    circuits/mosfet/channel height/.initial={0.32cm},
    circuits/mosfet/channel length/.initial={0.6cm},
    circuits/mosfet/arrow length/.initial={1.75mm},
    circuits/mosfet/arrow width/.initial={1.75mm},
    circuits/mosfet/gate line width/.initial={1.2pt},
    circuits/mosfet/bulk line width/.initial={0.8pt},
    circuits/mosfet/gate dot radius/.initial=1.5pt,
    circuits/mosfet/gate dot line width/.initial=0.8pt,
    % mosfet options
    circuits/mosfet/no source arrow/.is if=mosfetnoarrow,
    circuits/mosfet/fill gate dot/.is if=mosfetfilldot,
    circuits/mosfet/draw bulk/.is if=mosfetdrawbulk,
    circuits/mosfet/draw bulk arrow/.is if=mosfetdrawbulkarrow,
    circuits/mosfet/mirror/.is if=mosfetmirror,
}


\makeatletter
\pgfdeclareshape{nmos}
{
    \saveddimen{\gatewidth}{\pgf@x=\pgfkeysvalueof{/tikz/circuits/mosfet/gate width}}
    \saveddimen{\gateskip}{\pgf@x=\pgfkeysvalueof{/tikz/circuits/mosfet/gate skip}}
    \saveddimen{\drainlength}{\pgf@x=\pgfkeysvalueof{/tikz/circuits/mosfet/drain length}}
    \saveddimen{\sourcelength}{\pgf@x=\pgfkeysvalueof{/tikz/circuits/mosfet/source length}}
    \saveddimen{\channellength}{\pgf@x=\pgfkeysvalueof{/tikz/circuits/mosfet/channel length}}
    \saveddimen{\channelheight}{\pgf@x=\pgfkeysvalueof{/tikz/circuits/mosfet/channel height}}
    \saveddimen{\arrowlength}{\pgf@x=\pgfkeysvalueof{/tikz/circuits/mosfet/arrow length}}
    \saveddimen{\arrowwidth}{\pgf@x=\pgfkeysvalueof{/tikz/circuits/mosfet/arrow width}}

    % electrical terminals
    \anchor{drain}{\pgfpointadd{\pgfpointorigin}{\pgfpoint{0cm}{0.5 * \channellength + \drainlength}}}
    \anchor{source}{\pgfpointadd{\pgfpointorigin}{\pgfpoint{0cm}{-0.5 * \channellength - \sourcelength}}}
    \anchor{gate}{\pgfpointadd{\pgfpointorigin}{\pgfpoint{-\channelheight -\gateskip}{0cm}}}
    \anchor{bulk}{\pgfpointorigin}
    % general anchors
    \anchor{center}{\pgfpointadd{\pgfpointorigin}{\pgfpoint{(-\channelheight - \gateskip)/2}{0cm}}}
    \anchor{north}{\pgfpointadd{\pgfpointorigin}{\pgfpoint{(-\channelheight - \gateskip)/2}{0.5 * \channellength + \drainlength}}}
    \anchor{south}{\pgfpointadd{\pgfpointorigin}{\pgfpoint{(-\channelheight - \gateskip)/2}{-0.5 * \channellength - \drainlength}}}
    \anchor{west}{\pgfpointadd{\pgfpointorigin}{\pgfpoint{-\channelheight -\gateskip}{0cm}}}
    \anchor{east}{\pgfpointorigin}
    \anchor{north west}{\pgfpointadd{\pgfpointorigin}{\pgfpoint{-\channelheight - \gateskip}{0.5 * \channellength + \drainlength}}}
    \anchor{north east}{\pgfpointadd{\pgfpointorigin}{\pgfpoint{0cm}{0.5 * \channellength + \drainlength}}}
    \anchor{south west}{\pgfpointadd{\pgfpointorigin}{\pgfpoint{-\channelheight - \gateskip}{-0.5 * \channellength - \drainlength}}}
    \anchor{south east}{\pgfpointadd{\pgfpointorigin}{\pgfpoint{0cm}{-0.5 * \channellength - \drainlength}}}
    \beforebackgroundpath{
        \pgfsetbuttcap
        \pgfsetlinewidth{\pgfkeysvalueof{/tikz/circuits/line width}}
        %% draw channel, source and drain extensions
        % move to drain
        \pgfpathmoveto{\pgfpointadd{\pgfpointorigin}{\pgfpoint{0cm}{0.5 * \channellength + \drainlength}}}
        % draw drain extension
        \pgfpathlineto{\pgfpointadd{\pgfpointorigin}{\pgfpoint{0cm}{0.5 * \channellength}}}
        % draw to channel
        \pgfpathlineto{\pgfpointadd{\pgfpointorigin}{\pgfpoint{-\channelheight}{0.5 * \channellength}}}
        % draw half of channel
        \pgfpathlineto{\pgfpointadd{\pgfpointorigin}{\pgfpoint{-\channelheight}{0cm}}}
        % draw other half of channel (starting from source
        \pgfpathlineto{\pgfpointadd{\pgfpointorigin}{\pgfpoint{-\channelheight}{-0.5 * \channellength}}}
        % move to 'inner' source (without extension)
        \pgfpathlineto{\pgfpointadd{\pgfpointorigin}{\pgfpoint{0cm}{-0.5 * \channellength}}}
        % draw source extension
        \pgfpathlineto{\pgfpointadd{\pgfpointorigin}{\pgfpoint{0cm}{-0.5 * \channellength - \sourcelength}}}
        \pgfusepath{stroke}
        %% draw gate
        \pgfsetlinewidth{\pgfkeysvalueof{/tikz/circuits/mosfet/gate line width}}
        \pgfpathmoveto{\pgfpointadd{\pgfpointorigin}{\pgfpoint{-\channelheight - \gateskip}{0.5 * \channellength + 0.5 * \pgfkeysvalueof{/tikz/circuits/line width}}}}
        \pgfpathlineto{\pgfpointadd{\pgfpointorigin}{\pgfpoint{-\channelheight - \gateskip}{-0.5 * \channellength - 0.5 * \pgfkeysvalueof{/tikz/circuits/line width}}}}
        \pgfusepath{stroke}
        % draw bulk, source arrow and gate circle, depending on the display options
        \ifmosfetdrawbulk
            \ifmosfetnoarrow
                % draw bulk (with arrow)
                \pgfsetlinewidth{\pgfkeysvalueof{/tikz/circuits/mosfet/bulk line width}}
                \pgfpathmoveto{\pgfpointorigin}
                \pgfpathlineto{\pgfpointadd{\pgfpointorigin}{\pgfpoint{-\channelheight}{0cm}}}
                \pgfsetarrows{-{Stealth[length=\pgfkeysvalueof{/tikz/circuits/mosfet/arrow length}, width=\pgfkeysvalueof{/tikz/circuits/mosfet/arrow width}]}}
                \pgfusepath{stroke}
            \else
                % draw bulk (without arrow)
                \pgfsetlinewidth{\pgfkeysvalueof{/tikz/circuits/mosfet/bulk line width}}
                \pgfpathmoveto{\pgfpointorigin}
                \pgfpathlineto{\pgfpointadd{\pgfpointorigin}{\pgfpoint{-\channelheight}{0cm}}}
                \pgfusepath{stroke}
                % draw source arrow
                \pgfsetlinewidth{0.8*\pgfkeysvalueof{/tikz/circuits/line width}}
                \pgfpathmoveto{\pgfpointadd{\pgfpointorigin}{\pgfpoint{-\channelheight}{-0.5 * \channellength}}}
                \pgfpathlineto{\pgfpointadd{\pgfpointorigin}{\pgfpoint{0cm}{-0.5 * \channellength}}}
                \pgfsetarrows{-{Stealth[length=\pgfkeysvalueof{/tikz/circuits/mosfet/arrow length}, width=\pgfkeysvalueof{/tikz/circuits/mosfet/arrow width}]}}
                \pgfusepath{stroke}
                \pgfsetarrowsend{}
            \fi
        \else
            \ifmosfetnoarrow
                % no circle in nmos
            \else
                % draw source arrow
                \pgfsetlinewidth{0.8*\pgfkeysvalueof{/tikz/circuits/line width}}
                \pgfpathmoveto{\pgfpointadd{\pgfpointorigin}{\pgfpoint{-\channelheight}{-0.5 * \channellength}}}
                \pgfpathlineto{\pgfpointadd{\pgfpointorigin}{\pgfpoint{0cm}{-0.5 * \channellength}}}
                \pgfsetarrows{-{Stealth[length=\pgfkeysvalueof{/tikz/circuits/mosfet/arrow length}, width=\pgfkeysvalueof{/tikz/circuits/mosfet/arrow width}]}}
                \pgfusepath{stroke}
                \pgfsetarrowsend{}
            \fi
        \fi
    }
}
\makeatother

% vim: ft=plaintex nowrap

\newif\ifmosfetdrawbulk
\newif\ifmosfetdrawbulkarrow
\newif\ifmosfetmirror
\newif\ifmosfetnoarrow
\newif\ifmosfetfilldot

\tikzset{
    circuits/mosfet/scale/.initial=1, % not implemented
    circuits/mosfet/gate width/.initial={0.6cm},
    circuits/mosfet/gate skip/.initial={0.1cm},
    circuits/mosfet/drain length/.initial={0.2cm},
    circuits/mosfet/source length/.initial={0.2cm},
    circuits/mosfet/channel height/.initial={0.32cm},
    circuits/mosfet/channel length/.initial={0.6cm},
    circuits/mosfet/arrow length/.initial={1.75mm},
    circuits/mosfet/arrow width/.initial={1.75mm},
    circuits/mosfet/gate line width/.initial={1.2pt},
    circuits/mosfet/bulk line width/.initial={0.8pt},
    circuits/mosfet/gate dot radius/.initial=1.5pt,
    circuits/mosfet/gate dot line width/.initial=0.8pt,
    % mosfet options
    circuits/mosfet/no source arrow/.is if=mosfetnoarrow,
    circuits/mosfet/fill gate dot/.is if=mosfetfilldot,
    circuits/mosfet/draw bulk/.is if=mosfetdrawbulk,
    circuits/mosfet/draw bulk arrow/.is if=mosfetdrawbulkarrow,
    circuits/mosfet/mirror/.is if=mosfetmirror,
}


\newif\ifpmosfillgatecircle

\tikzset{
    circuits/pmos/fill gate circle/.is if=pmosfillgatecircle
}

\makeatletter
\pgfdeclareshape{pmos}
{
    \saveddimen{\gatewidth}{\pgf@x=\pgfkeysvalueof{/tikz/circuits/mosfet/gate width}}
    \saveddimen{\gateskip}{\pgf@x=\pgfkeysvalueof{/tikz/circuits/mosfet/gate skip}}
    \saveddimen{\drainlength}{\pgf@x=\pgfkeysvalueof{/tikz/circuits/mosfet/drain length}}
    \saveddimen{\sourcelength}{\pgf@x=\pgfkeysvalueof{/tikz/circuits/mosfet/source length}}
    \saveddimen{\channellength}{\pgf@x=\pgfkeysvalueof{/tikz/circuits/mosfet/channel length}}
    \saveddimen{\channelheight}{\pgf@x=\pgfkeysvalueof{/tikz/circuits/mosfet/channel height}}
    \saveddimen{\arrowlength}{\pgf@x=\pgfkeysvalueof{/tikz/circuits/mosfet/arrow length}}
    \saveddimen{\arrowwidth}{\pgf@x=\pgfkeysvalueof{/tikz/circuits/mosfet/arrow width}}

    % electrical terminals
    \anchor{drain}{\pgfpointadd{\pgfpointorigin}{\pgfpoint{0cm}{-0.5 * \channellength - \drainlength}}}
    \anchor{source}{\pgfpointadd{\pgfpointorigin}{\pgfpoint{0cm}{0.5 * \channellength + \sourcelength}}}
    \anchor{gate}{\pgfpointadd{\pgfpointorigin}{\pgfpoint{-\channelheight -\gateskip}{0cm}}}
    \anchor{bulk}{\pgfpointorigin}
    % general anchors
    \anchor{center}{\pgfpointadd{\pgfpointorigin}{\pgfpoint{(-\channelheight - \gateskip)/2}{0cm}}}
    \anchor{north}{\pgfpointadd{\pgfpointorigin}{\pgfpoint{(-\channelheight - \gateskip)/2}{0.5 * \channellength + \drainlength}}}
    \anchor{south}{\pgfpointadd{\pgfpointorigin}{\pgfpoint{(-\channelheight - \gateskip)/2}{-0.5 * \channellength - \drainlength}}}
    \anchor{west}{\pgfpointadd{\pgfpointorigin}{\pgfpoint{-\channelheight -\gateskip}{0cm}}}
    \anchor{east}{\pgfpointorigin}
    \anchor{north west}{\pgfpointadd{\pgfpointorigin}{\pgfpoint{-\channelheight - \gateskip}{0.5 * \channellength + \drainlength}}}
    \anchor{north east}{\pgfpointadd{\pgfpointorigin}{\pgfpoint{0cm}{0.5 * \channellength + \drainlength}}}
    \anchor{south west}{\pgfpointadd{\pgfpointorigin}{\pgfpoint{-\channelheight - \gateskip}{-0.5 * \channellength - \drainlength}}}
    \anchor{south east}{\pgfpointadd{\pgfpointorigin}{\pgfpoint{0cm}{-0.5 * \channellength - \drainlength}}}
    \beforebackgroundpath{
        \pgfsetbuttcap
        \pgfsetlinewidth{\pgfkeysvalueof{/tikz/circuits/line width}}
        %% draw channel, source and drain extensions
        % move to source
        \pgfpathmoveto{\pgfpointadd{\pgfpointorigin}{\pgfpoint{0cm}{0.5 * \channellength + \sourcelength}}}
        % draw drain extension
        \pgfpathlineto{\pgfpointadd{\pgfpointorigin}{\pgfpoint{0cm}{0.5 * \channellength}}}
        % draw to channel
        \pgfpathlineto{\pgfpointadd{\pgfpointorigin}{\pgfpoint{-\channelheight}{0.5 * \channellength}}}
        % draw half of channel
        \pgfpathlineto{\pgfpointadd{\pgfpointorigin}{\pgfpoint{-\channelheight}{0cm}}}
        % draw other half of channel (starting from drain
        \pgfpathlineto{\pgfpointadd{\pgfpointorigin}{\pgfpoint{-\channelheight}{-0.5 * \channellength}}}
        % move to 'inner' source (without extension)
        \pgfpathlineto{\pgfpointadd{\pgfpointorigin}{\pgfpoint{0cm}{-0.5 * \channellength}}}
        % draw drain extension
        \pgfpathlineto{\pgfpointadd{\pgfpointorigin}{\pgfpoint{0cm}{-0.5 * \channellength - \drainlength}}}
        \pgfusepath{stroke}
        %% draw gate
        \pgfsetlinewidth{\pgfkeysvalueof{/tikz/circuits/mosfet/gate line width}}
        \pgfpathmoveto{\pgfpointadd{\pgfpointorigin}{\pgfpoint{-\channelheight - \gateskip}{0.5 * \channellength + 0.5 * \pgfkeysvalueof{/tikz/circuits/line width}}}}
        \pgfpathlineto{\pgfpointadd{\pgfpointorigin}{\pgfpoint{-\channelheight - \gateskip}{-0.5 * \channellength - 0.5 * \pgfkeysvalueof{/tikz/circuits/line width}}}}
        \pgfusepath{stroke}
        % draw bulk, source arrow and gate circle, depending on the display options
        \ifmosfetdrawbulk
            \ifmosfetnoarrow
                % draw bulk (with arrow)
                \pgfsetlinewidth{\pgfkeysvalueof{/tikz/circuits/mosfet/bulk line width}}
                \pgfpathmoveto{\pgfpointadd{\pgfpointorigin}{\pgfpoint{-\channelheight}{0cm}}}
                \pgfpathlineto{\pgfpointorigin}
                \pgfsetarrows{-{Stealth[length=\pgfkeysvalueof{/tikz/circuits/mosfet/arrow length}, width=\pgfkeysvalueof{/tikz/circuits/mosfet/arrow width}]}}
                \pgfusepath{stroke}
            \else
                % draw bulk (without arrow)
                \pgfsetlinewidth{\pgfkeysvalueof{/tikz/circuits/mosfet/bulk line width}}
                \pgfpathmoveto{\pgfpointorigin}
                \pgfpathlineto{\pgfpointadd{\pgfpointorigin}{\pgfpoint{-\channelheight}{0cm}}}
                \pgfusepath{stroke}
                % draw source arrow
                \pgfsetlinewidth{0.8*\pgfkeysvalueof{/tikz/circuits/line width}}
                \pgfpathmoveto{\pgfpointadd{\pgfpointorigin}{\pgfpoint{0cm}{0.5 * \channellength}}}
                \pgfpathlineto{\pgfpointadd{\pgfpointorigin}{\pgfpoint{-\channelheight}{0.5 * \channellength}}}
                \pgfsetarrows{-{Stealth[length=\pgfkeysvalueof{/tikz/circuits/mosfet/arrow length}, width=\pgfkeysvalueof{/tikz/circuits/mosfet/arrow width}]}}
                \pgfusepath{stroke}
                \pgfsetarrowsend{}
            \fi
        \else
            \ifmosfetnoarrow
                \pgfpathcircle{\pgfpointadd{\pgfpointorigin}{\pgfpoint{-\channelheight - \gateskip - \pgfkeysvalueof{/tikz/circuits/mosfet/gate dot radius}}{0cm}}}{\pgfkeysvalueof{/tikz/circuits/mosfet/gate dot radius}}
                \ifpmosfillgatecircle
                    \pgfusepath{stroke, fill}
                \else
                    \pgfusepath{stroke}
                \fi
            \else
                \pgfsetlinewidth{0.8*\pgfkeysvalueof{/tikz/circuits/line width}}
                \pgfpathmoveto{\pgfpointadd{\pgfpointorigin}{\pgfpoint{0cm}{0.5 * \channellength}}}
                \pgfpathlineto{\pgfpointadd{\pgfpointorigin}{\pgfpoint{-\channelheight}{0.5 * \channellength}}}
                \pgfsetarrows{-{Stealth[length=\pgfkeysvalueof{/tikz/circuits/mosfet/arrow length}, width=\pgfkeysvalueof{/tikz/circuits/mosfet/arrow width}]}}
                \pgfusepath{stroke}
                \pgfsetarrowsend{}
            \fi
        \fi
    }
}
\makeatother

% vim: ft=plaintex nowrap

\input{symbols/current_source}
\input{symbols/ground}
\input{symbols/vdd}
\input{symbols/impedance}
\tikzset{
    circuits/amplifier/width/.initial = 1.25cm,
    circuits/amplifier/height/.initial = 1.25cm,
    amplifier/.style = {amplifiershape}
}

\makeatletter
\pgfdeclareshape{amplifiershape}
{
    \saveddimen{\width}{\pgf@x=\pgfkeysvalueof{/tikz/circuits/amplifier/width}}
    \saveddimen{\inputheight}{\pgf@x=\pgfkeysvalueof{/tikz/circuits/amplifier/height}}
    \savedanchor{\centerpoint}{\pgfpointorigin}
    \savedanchor{\inputplus}{%
        \pgfpointadd%
        {\pgfpointorigin}%
        {\pgfpoint%
            {-0.5 * \pgfkeysvalueof{/tikz/circuits/amplifier/width}}%
            {0.5 * \pgfkeysvalueof{/tikz/circuits/amplifier/height}}
        }%
    }
    \savedanchor{\inputminus}{%
        \pgfpointadd%
        {\pgfpointorigin}%
        {\pgfpoint%
            {-0.5 * \pgfkeysvalueof{/tikz/circuits/amplifier/width}}%
            {-0.5 * \pgfkeysvalueof{/tikz/circuits/amplifier/height}}
        }%
    }
    \savedanchor{\output}{%
        \pgfpointadd%
        {\pgfpointorigin}%
        {\pgfpoint%
            {0.5 * \pgfkeysvalueof{/tikz/circuits/amplifier/width}}%
            {0cm}
        }%
    }
    % electrical terminals (anchors)
    \anchor{in}{\pgfpointadd{\centerpoint}{\pgfpoint{-\width/2}{0cm}}}
    \anchor{out}{\output}
    \anchor{+power}{
        \pgfpointlineattime%
        {0.5}%
        {\pgfpointadd{\inputplus}{\pgfpoint{0cm}{0.5 * \inputheight}}}
        {\output}%
    }
    \anchor{-power}{
        \pgfpointlineattime%
        {0.5}%
        {\pgfpointadd{\inputminus}{\pgfpoint{0cm}{-0.5 * \inputheight}}}
        {\output}%
    }
    % regular anchors
    \anchor{center}{\centerpoint}
    \anchor{north}{
        \pgfpointadd{\inputplus}{\pgfpoint{\width/2}{0.5 * \inputheight}}
    }
    \anchor{south}{
        \pgfpointadd{\inputminus}{\pgfpoint{\width/2}{-0.5 * \inputheight}}
    }
    \anchor{east}{
        \pgfpointadd{\pgfpointorigin}{\pgfpoint{\width/2}{0cm}}
    }
    \anchor{west}{
        \pgfpointadd{\pgfpointorigin}{\pgfpoint{-\width/2}{0cm}}
    }
    \anchor{north west}{
        \pgfpointadd{\inputplus}{\pgfpoint{0cm}{0.5 * \inputheight}}
    }
    \anchor{south west}{
        \pgfpointadd{\inputminus}{\pgfpoint{0cm}{-0.5 * \inputheight}}
    }
    \anchor{north east}{
        \pgfpointadd{\inputplus}{\pgfpoint{\width}{0.5 * \inputheight}}
    }
    \anchor{south east}{
        \pgfpointadd{\inputminus}{\pgfpoint{\width}{-0.5 * \inputheight}}
    }
    \backgroundpath{
        \pgfsetlinewidth{\pgfkeysvalueof{/tikz/circuits/line width}}
        \pgfpathmoveto{\inputplus}
        \pgfpathlineto{\output}
        \pgfpathlineto{\inputminus}
        \pgfpathclose
        \pgfusepath{stroke}
    }
}
\makeatother

% vim: ft=plaintex nowrap

\tikzset{
    circuits/mixer/radius/.initial = 0.5cm,
    mixer/.style = {mixershape}
}

\makeatletter
\pgfdeclareshape{mixershape}
{
    \saveddimen{\radius}{\pgf@x=\pgfkeysvalueof{/tikz/circuits/mixer/radius}}

    % general anchors
    \anchor{center}{\pgfpointorigin}
    \anchor{west}{\pgfpointadd{\pgfpointorigin}{\pgfpoint{-\radius}{0cm}}}
    \anchor{east}{\pgfpointadd{\pgfpointorigin}{\pgfpoint{\radius}{0cm}}}
    \anchor{north}{\pgfpointadd{\pgfpointorigin}{\pgfpoint{0cm}{ \radius}}}
    \anchor{south}{\pgfpointadd{\pgfpointorigin}{\pgfpoint{0cm}{-\radius}}}
    \anchor{south east}{\pgfpointadd{\pgfpointorigin}{\pgfpoint{ \radius}{-\radius}}}
    \anchor{south west}{\pgfpointadd{\pgfpointorigin}{\pgfpoint{-\radius}{-\radius}}}
    \anchor{north east}{\pgfpointadd{\pgfpointorigin}{\pgfpoint{ \radius}{ \radius}}}
    \anchor{north west}{\pgfpointadd{\pgfpointorigin}{\pgfpoint{-\radius}{ \radius}}}
    \beforebackgroundpath{
        \pgfsetlinewidth{\pgfkeysvalueof{/tikz/circuits/line width}}
        \pgfpathcircle{\pgfpointorigin}{\radius}
        \pgfusepath{stroke}
        % draw cross
        \pgfpathmoveto{\pgfpointadd{\pgfpointorigin}{\pgfpoint{-0.707107 * \radius}{-0.707107 * \radius}}}
        \pgfpathlineto{\pgfpointadd{\pgfpointorigin}{\pgfpoint{ 0.707107 * \radius}{ 0.707107 * \radius}}}
        \pgfpathmoveto{\pgfpointadd{\pgfpointorigin}{\pgfpoint{-0.707107 * \radius}{ 0.707107 * \radius}}}
        \pgfpathlineto{\pgfpointadd{\pgfpointorigin}{\pgfpoint{ 0.707107 * \radius}{-0.707107 * \radius}}}
        \pgfusepath{stroke}
    }
}
\makeatother

% vim: ft=plaintex

\input{symbols/oscillator}
\input{symbols/antenna}
\input{symbols/gyrator}
\input{symbols/inverter}

% vim: ft=plaintex nowrap
