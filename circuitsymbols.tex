\newif\ifcircuitsflip

% drawing styles
\tikzset{
    wirethickness/.style = {thick},
    wire/.style={wirethickness},
    arrowwire/.style = {
        wirethickness,
        ->,
        >=stealth,
    },
    lwire/.style={wirethickness, line cap = rect},
    scriptsize/.style={font=\scriptsize},
    ---/.style args={#1 and #2}{to path={-- ++(#1, 0) |- ($(\tikztotarget)+(#2, 0)$)}},
    ---/.default = 0cm and 0cm,
    -|-/.style args={#1 and #2}{to path={-| ([xshift=#1, yshift=#2]$(\tikztostart)!0.5!(\tikztotarget)$) |- (\tikztotarget) \tikztonodes}},
    -|-/.default = 0cm and 0cm,
    |-|/.style args={#1 and #2}{to path={|- ([xshift=#1, yshift=#2]$(\tikztostart)!0.5!(\tikztotarget)$) -| (\tikztotarget) \tikztonodes}},
    |-|/.default = 0cm and 0cm,
    angleconnect/.style={to path={-- ++(#1, 0) -- ($(\tikztotarget)-(#1, 0)$) -- (\tikztotarget)}},
    angleconnect/.value required = true,
    tmodule/.style args ={#1 and #2}{draw, rectangle, wire, minimum width = #1, minimum height = #2},
    tmodule/.default = {0cm and 0cm},
    module/.style args ={#1 and #2}{
        wire,
        draw, rectangle,
        align = center,
        minimum width = #1,
        minimum height = #2
    },
    module/.default = 1cm and 1cm,
    sumpoint/.style = {
        wire,
        draw, circle,
        minimum size = 0.5cm,
        inner sep = 0cm
    },
    current arrow/.style = {
        postaction = decorate,
        decoration = {
            markings,
            mark = at position #1 with {\arrow{stealth}}
        }
    },
    current arrow/.default = 0.5,
    current arrow reversed/.style = {
        postaction = decorate,
        decoration = {
            markings,
            mark = at position #1 with {\arrowreversed{stealth}}
        }
    },
    current arrow reversed/.default = 0.5,
    % flip components (effect depends on shape)
    flip/.is if = circuitsflip
}

\pgfkeys{
    % general circuits options
    /tikz/circuits/line width/.initial = 0.8pt,
    /tikz/circuits/dot/radius/.initial = 0.05cm,
    /tikz/dot/.style = {dotshape},
    /tikz/circuits/port/radius/.initial = 0.05cm,
    %/tikz/port/.style = {portshape}
    /tikz/port/.style = {wire, draw=black, fill=white, circle, inner sep = 0cm, minimum size = 0.1cm}
}

\makeatletter
\pgfdeclareshape{dotshape}
{
    \saveddimen{\radius}{\pgf@x=\pgfkeysvalueof{/tikz/circuits/dot/radius}}
    \anchor{center}{\pgfpointorigin}
    \anchor{north}{\pgfpointorigin}
    \anchor{south}{\pgfpointorigin}
    \anchor{west}{\pgfpointorigin}
    \anchor{east}{\pgfpointorigin}
    \backgroundpath{
        \pgfpathcircle{\pgfpointorigin}{\radius}
        \pgfusepath{fill}
    }
}
\pgfdeclareshape{portshape}
{
    \saveddimen{\radius}{\pgf@x=\pgfkeysvalueof{/tikz/circuits/port/radius}}
    \anchor{center}{\pgfpointorigin}
    \backgroundpath{
        \color{white}
        \pgfsetstrokecolor{black}
        \pgfsetlinewidth{\pgfkeysvalueof{/tikz/circuits/line width}}
        \pgfpathcircle{\pgfpointorigin}{\radius}
        \pgfusepath{fill, stroke}
    }
}
\makeatother

\newif\ifotaflipinputs
\newif\ifotaflipoutputs
\newif\ifotafullydifferential
\newif\ifotanopinlabels

\tikzset{
    circuits/ota/scale/.initial = 1,
    circuits/ota/width/.initial = 2cm,
    circuits/ota/input height/.initial = 2cm,
    circuits/ota/output height/.initial = 1cm,
    circuits/ota/input distribution factor/.initial = 0.6,
    circuits/ota/output distribution factor/.initial = 0.8,
    circuits/ota/pin fontsize/.initial = {\small},
    circuits/ota/flip inputs/.is if=otaflipinputs,
    circuits/ota/flip outputs/.is if=otaflipoutputs,
    circuits/ota/fully differential/.is if=otafullydifferential,
    circuits/ota/no pin labels/.is if=otanopinlabels,
    ota/.style = {fdotashape, circuits/ota/fully differential=false},
    fdota/.style = {fdotashape, circuits/ota/fully differential=true}
}

\makeatletter
\pgfdeclareshape{fdotashape}
{
    \saveddimen{\width}{\pgf@x=\pgfkeysvalueof{/tikz/circuits/ota/width}}
    \saveddimen{\inputheight}{\pgf@x=\pgfkeysvalueof{/tikz/circuits/ota/input height}}
    \saveddimen{\outputheight}{\pgf@x=\pgfkeysvalueof{/tikz/circuits/ota/output height}}
    \savedmacro{\inputdistribution}{\renewcommand{\inputdistribution}[0]{\pgfkeysvalueof{/tikz/circuits/ota/input distribution factor}}}
    \savedmacro{\outputdistribution}{\renewcommand{\outputdistribution}[0]{\pgfkeysvalueof{/tikz/circuits/ota/output distribution factor}}}
    \savedmacro{\scale}{\renewcommand{\scale}[0]{\pgfkeysvalueof{/tikz/circuits/ota/scale}}}
    %\savedmacro{\pinfontsize}{\renewcommand{\pinfontsize}[0]{\pgfkeysvalueof{/tikz/circuits/ota/pin fontsize}}}
    \savedanchor{\centerpoint}{\pgfpointorigin}
    \savedanchor{\inputplus}{%
        \pgfpointadd%
        {\pgfpointorigin}%
        {\pgfpoint%
            {-0.5 * \pgfkeysvalueof{/tikz/circuits/ota/width}}%
            {0.5 * \pgfkeysvalueof{/tikz/circuits/ota/input height} * \pgfkeysvalueof{/tikz/circuits/ota/input distribution factor}}%
        }%
    }
    \savedanchor{\inputminus}{%
        \pgfpointadd%
        {\pgfpointorigin}%
        {\pgfpoint%
            {-0.5 * \pgfkeysvalueof{/tikz/circuits/ota/width}}%
            {-0.5 * \pgfkeysvalueof{/tikz/circuits/ota/input height} * \pgfkeysvalueof{/tikz/circuits/ota/input distribution factor}}%
        }%
    }
    \savedanchor{\outputminus}{%
        \pgfpointadd%
        {\pgfpointorigin}%
        {\pgfpoint%
            {0.5 * \pgfkeysvalueof{/tikz/circuits/ota/width}}%
            {0.5 * \pgfkeysvalueof{/tikz/circuits/ota/output height}}
        }%
    }
    \savedanchor{\outputplus}{%
        \pgfpointadd%
        {\pgfpointorigin}%
        {\pgfpoint%
            {0.5 * \pgfkeysvalueof{/tikz/circuits/ota/width}}%
            {0.5 * -\pgfkeysvalueof{/tikz/circuits/ota/output height}}
        }%
    }
    % electrical terminals (anchors)
    \anchor{cmin}{\pgfpointadd{\centerpoint}{\pgfpoint{-\width/2}{0cm}}}
    \anchor{cmout}{\pgfpointadd{\centerpoint}{\pgfpoint{\width/2}{0cm}}}
    \anchor{+in}{\inputplus}
    \anchor{-in}{\inputminus}
    \anchor{in}{\pgfpointadd{\centerpoint}{\pgfpoint{-\width/2}{0cm}}}
    \anchor{out}{
        \pgfpointadd%
        {\pgfpointorigin}%
        {\pgfpoint%
            {0.5 * \pgfkeysvalueof{/tikz/circuits/ota/width}}%
            {0cm}
        }%
    }
    \anchor{+out}{
        \pgfpointlineattime%
        {\outputdistribution}%
        {\pgfpointadd{\inputminus}{\pgfpoint{0cm}{-0.5 * \inputheight * (1 - \inputdistribution)}}}%
        {\outputplus}%
    }
    \anchor{-out}{
        \pgfpointlineattime%
        {\outputdistribution}%
        {\pgfpointadd{\inputplus}{\pgfpoint{0cm}{0.5 * \inputheight * (1 - \inputdistribution)}}}%
        {\outputminus}%
    }
    \anchor{+power}{
        \pgfpointlineattime%
        {0.5}%
        {\pgfpointadd{\inputplus}{\pgfpoint{0cm}{0.5 * \inputheight * (1 - \inputdistribution)}}}%
        {\outputminus}
    }
    \anchor{-power}{
        \pgfpointlineattime%
        {0.5}%
        {\pgfpointadd{\inputminus}{\pgfpoint{0cm}{-0.5 * \inputheight * (1 - \inputdistribution)}}}%
        {\outputplus}
    }
    % regular anchors
    \anchor{center}{\centerpoint}
    \anchor{north}{
        \pgfpointadd{\inputplus}{\pgfpoint{\width/2}{0.5 * \inputheight * (1 - \inputdistribution)}}
    }
    \anchor{south}{
        \pgfpointadd{\inputminus}{\pgfpoint{\width/2}{-0.5 * \inputheight * (1 - \inputdistribution)}}
    }
    \anchor{north west}{
        \pgfpointadd{\inputplus}{\pgfpoint{0cm}{0.5 * \inputheight * (1 - \inputdistribution)}}
    }
    \anchor{south west}{
        \pgfpointadd{\inputminus}{\pgfpoint{0cm}{-0.5 * \inputheight * (1 - \inputdistribution)}}
    }
    \anchor{north east}{
        \pgfpointadd{\inputplus}{\pgfpoint{\width}{0.5 * \inputheight * (1 - \inputdistribution)}}
    }
    \anchor{south east}{
        \pgfpointadd{\inputminus}{\pgfpoint{\width}{-0.5 * \inputheight * (1 - \inputdistribution)}}
    }
    \backgroundpath{
        \pgfsetlinewidth{\pgfkeysvalueof{/tikz/circuits/line width}}
        \pgfpathmoveto{\pgfpointadd{\inputplus}{\pgfpoint{0cm}{0.5 * \inputheight * (1 - \inputdistribution)}}}
        \pgfpathlineto{\outputminus}
        \pgfpathlineto{\outputplus}
        \pgfpathlineto{\pgfpointadd{\inputminus}{\pgfpoint{0cm}{-0.5 * \inputheight * (1 - \inputdistribution)}}}
        \pgfpathclose
        \pgfusepath{stroke}
        \ifotanopinlabels
        \else
            \ifotaflipinputs
                \pgftext[left, at={\inputplus}, x=2pt]{\pgfkeysvalueof{/tikz/circuits/ota/pin fontsize}$-$}
                \pgftext[left, at={\inputminus}, x=2pt]{\pgfkeysvalueof{/tikz/circuits/ota/pin fontsize}$+$}
            \else
                \pgftext[left, at={\inputplus}, x=2pt]{\pgfkeysvalueof{/tikz/circuits/ota/pin fontsize}$+$}
                \pgftext[left, at={\inputminus}, x=2pt]{\pgfkeysvalueof{/tikz/circuits/ota/pin fontsize}$-$}
            \fi
            \ifotafullydifferential
                \ifotaflipoutputs
                    \pgftext[at={\pgfpointlineattime{\outputdistribution}{\pgfpointadd{\inputplus}{\pgfpoint{0cm}{0.5 * \inputheight * (1 - \inputdistribution)}}}{\outputminus} }, x = -3pt, y=-6pt]{\pgfkeysvalueof{/tikz/circuits/ota/pin fontsize}$+$}
                    \pgftext[at={\pgfpointlineattime{\outputdistribution}{\pgfpointadd{\inputminus}{\pgfpoint{0cm}{-0.5 * \inputheight * (1 - \inputdistribution)}}}{\outputplus} }, x = -3pt, y=6pt]{\pgfkeysvalueof{/tikz/circuits/ota/pin fontsize}$-$}
                \else
                    \pgftext[at={\pgfpointlineattime{\outputdistribution}{\pgfpointadd{\inputplus}{\pgfpoint{0cm}{0.5 * \inputheight * (1 - \inputdistribution)}}}{\outputminus} }, x = -3pt, y=-6pt]{\pgfkeysvalueof{/tikz/circuits/ota/pin fontsize}$-$}
                    \pgftext[at={\pgfpointlineattime{\outputdistribution}{\pgfpointadd{\inputminus}{\pgfpoint{0cm}{-0.5 * \inputheight * (1 - \inputdistribution)}}}{\outputplus} }, x = -3pt, y=6pt]{\pgfkeysvalueof{/tikz/circuits/ota/pin fontsize}$+$}
                \fi
            \fi
        \fi
    }
}
\makeatother

% vim: ft=plaintex nowrap

\newif\ifopampflipinputs
\newif\ifopampflipoutputs
\newif\ifopampfullydifferential

\tikzset{
    circuits/opamp/width/.initial = 2cm,
    circuits/opamp/height/.initial = 2cm,
    circuits/opamp/input distribution factor/.initial = 0.6,
    circuits/opamp/output distribution factor/.initial = 0.6,
    circuits/opamp/pin fontsize/.initial = {\tiny},
    circuits/opamp/flip inputs/.is if=opampflipinputs,
    circuits/opamp/flip outputs/.is if=opampflipoutputs,
    circuits/opamp/fully differential/.is if=opampfullydifferential,
    opamp/.style = {fdopampshape, circuits/opamp/fully differential=false},
    fdopamp/.style = {fdopampshape, circuits/opamp/fully differential=true}
}

\makeatletter
\pgfdeclareshape{fdopampshape}
{
    \saveddimen{\width}{\pgf@x=\pgfkeysvalueof{/tikz/circuits/opamp/width}}
    \saveddimen{\inputheight}{\pgf@x=\pgfkeysvalueof{/tikz/circuits/opamp/height}}
    \savedmacro{\inputdistribution}{\renewcommand{\inputdistribution}[0]{\pgfkeysvalueof{/tikz/circuits/opamp/input distribution factor}}}
    \savedmacro{\outputdistribution}{\renewcommand{\outputdistribution}[0]{\pgfkeysvalueof{/tikz/circuits/opamp/output distribution factor}}}
    %\savedmacro{\pinfontsize}{\renewcommand{\pinfontsize}[0]{\pgfkeysvalueof{/tikz/circuits/opamp/pin fontsize}}}
    \savedanchor{\centerpoint}{\pgfpointorigin}
    \savedanchor{\inputplus}{%
        \pgfpointadd%
        {\pgfpointorigin}%
        {\pgfpoint%
            {-0.5 * \pgfkeysvalueof{/tikz/circuits/opamp/width}}%
            {0.5 * \pgfkeysvalueof{/tikz/circuits/opamp/height} * \pgfkeysvalueof{/tikz/circuits/opamp/input distribution factor}}%
        }%
    }
    \savedanchor{\inputminus}{%
        \pgfpointadd%
        {\pgfpointorigin}%
        {\pgfpoint%
            {-0.5 * \pgfkeysvalueof{/tikz/circuits/opamp/width}}%
            {-0.5 * \pgfkeysvalueof{/tikz/circuits/opamp/height} * \pgfkeysvalueof{/tikz/circuits/opamp/input distribution factor}}%
        }%
    }
    \savedanchor{\output}{%
        \pgfpointadd%
        {\pgfpointorigin}%
        {\pgfpoint%
            {0.5 * \pgfkeysvalueof{/tikz/circuits/opamp/width}}%
            {0cm}
        }%
    }
    % electrical terminals (anchors)
    \anchor{cmin}{\pgfpointadd{\centerpoint}{\pgfpoint{-\width/2}{0cm}}}
    \anchor{cmout}{\output}
    \anchor{out}{\output}
    \anchor{+in}{\inputplus}
    \anchor{-in}{\inputminus}
    \anchor{+out}{
        \pgfpointlineattime%
        {\outputdistribution}%
        {\pgfpointadd{\inputminus}{\pgfpoint{0cm}{-0.5 * \inputheight * (1 - \inputdistribution)}}}%
        {\output}%
    }
    \anchor{-out}{
        \pgfpointlineattime%
        {\outputdistribution}%
        {\pgfpointadd{\inputplus}{\pgfpoint{0cm}{0.5 * \inputheight * (1 - \inputdistribution)}}}%
        {\output}%
    }
    \anchor{+power}{
        \pgfpointlineattime%
        {0.5}%
        {\pgfpointadd{\inputplus}{\pgfpoint{0cm}{0.5 * \inputheight * (1 - \inputdistribution)}}}%
        {\output}%
    }
    \anchor{-power}{
        \pgfpointlineattime%
        {0.5}%
        {\pgfpointadd{\inputminus}{\pgfpoint{0cm}{-0.5 * \inputheight * (1 - \inputdistribution)}}}%
        {\output}%
    }
    % regular anchors
    \anchor{center}{\centerpoint}
    \anchor{north}{
        \pgfpointadd{\inputplus}{\pgfpoint{\width/2}{0.5 * \inputheight * (1 - \inputdistribution)}}
    }
    \anchor{south}{
        \pgfpointadd{\inputminus}{\pgfpoint{\width/2}{-0.5 * \inputheight * (1 - \inputdistribution)}}
    }
    \anchor{north west}{
        \pgfpointadd{\inputplus}{\pgfpoint{0cm}{0.5 * \inputheight * (1 - \inputdistribution)}}
    }
    \anchor{south west}{
        \pgfpointadd{\inputminus}{\pgfpoint{0cm}{-0.5 * \inputheight * (1 - \inputdistribution)}}
    }
    \anchor{north east}{
        \pgfpointadd{\inputplus}{\pgfpoint{\width}{0.5 * \inputheight * (1 - \inputdistribution)}}
    }
    \anchor{south east}{
        \pgfpointadd{\inputminus}{\pgfpoint{\width}{-0.5 * \inputheight * (1 - \inputdistribution)}}
    }
    \backgroundpath{
        \pgfsetlinewidth{\pgfkeysvalueof{/tikz/circuits/line width}}
        \pgfpathmoveto{\pgfpointadd{\inputplus}{\pgfpoint{0cm}{0.5 * \inputheight * (1 - \inputdistribution)}}}
        \pgfpathlineto{\output}
        \pgfpathlineto{\pgfpointadd{\inputminus}{\pgfpoint{0cm}{-0.5 * \inputheight * (1 - \inputdistribution)}}}
        \pgfpathclose
        \pgfusepath{stroke}
        \ifopampflipinputs
            \pgftext[left, at={\inputplus}, x=2pt]{\tiny$-$}
            \pgftext[left, at={\inputminus}, x=2pt]{\tiny$+$}
        \else
            \pgftext[left, at={\inputplus}, x=2pt]{\tiny$+$}
            \pgftext[left, at={\inputminus}, x=2pt]{\tiny$-$}
        \fi
        \ifopampfullydifferential
            \ifopampflipoutputs
                \pgftext[at={\pgfpointlineattime{\outputdistribution}{\pgfpointadd{\inputplus}{\pgfpoint{0cm}{0.5 * \inputheight * (1 - \inputdistribution)}}}{\output} }, x = -3pt, y=-3pt]{\tiny$+$}
                \pgftext[at={\pgfpointlineattime{\outputdistribution}{\pgfpointadd{\inputminus}{\pgfpoint{0cm}{-0.5 * \inputheight * (1 - \inputdistribution)}}}{\output} }, x = -3pt, y=3pt]{\tiny$-$}
            \else
                \pgftext[at={\pgfpointlineattime{\outputdistribution}{\pgfpointadd{\inputplus}{\pgfpoint{0cm}{0.5 * \inputheight * (1 - \inputdistribution)}}}{\output} }, x = -3pt, y=-3pt]{\tiny$-$}
                \pgftext[at={\pgfpointlineattime{\outputdistribution}{\pgfpointadd{\inputminus}{\pgfpoint{0cm}{-0.5 * \inputheight * (1 - \inputdistribution)}}}{\output} }, x = -3pt, y=3pt]{\tiny$+$}
            \fi
        \fi
    }
}
\makeatother

% vim: ft=plaintex nowrap

\tikzset{
    circuits/capacitor/width/.initial = 0.1cm,
    circuits/capacitor/height/.initial = 0.5cm,
    capacitor/.style = {capacitorshape, anchor = plus}
}

\makeatletter
\pgfdeclareshape{capacitorshape}
{
    \saveddimen{\width}{\pgf@x=\pgfkeysvalueof{/tikz/circuits/capacitor/width}}
    \saveddimen{\height}{\pgf@x=\pgfkeysvalueof{/tikz/circuits/capacitor/height}}
    \savedanchor{\plus}{
        \pgfpointadd{\pgfpointorigin}{\pgfpoint{-0.5 * \pgfkeysvalueof{/tikz/circuits/capacitor/width}}{0cm}}
    }
    \savedanchor{\minus}{
        \pgfpointadd{\pgfpointorigin}{\pgfpoint{0.5 * \pgfkeysvalueof{/tikz/circuits/capacitor/width}}{0cm}}
    }

    % electrical terminals
    \anchor{plus}{\plus}
    \anchor{minus}{\minus}
    % general anchors
    \anchor{center}{\pgfpointorigin}
    \anchor{east}{\minus}
    \anchor{west}{\plus}
    \anchor{north}{\pgfpointadd{\pgfpointorigin}{\pgfpoint{0cm}{0.5 * \height}}}
    \anchor{south}{\pgfpointadd{\pgfpointorigin}{\pgfpoint{0cm}{-0.5 * \height}}}
    \anchor{south east}{\pgfpointadd{\pgfpointorigin}{\pgfpoint{0.5 * \height}{-0.5 * \width}}}
    \anchor{south west}{\pgfpointadd{\pgfpointorigin}{\pgfpoint{-0.5 * \height}{-0.5 * \width}}}
    \anchor{north east}{\pgfpointadd{\pgfpointorigin}{\pgfpoint{0.5 * \height}{0.5 * \width}}}
    \anchor{north west}{\pgfpointadd{\pgfpointorigin}{\pgfpoint{-0.5 * \height}{0.5 * \width}}}
    \beforebackgroundpath{
        \pgfsetlinewidth{\pgfkeysvalueof{/tikz/circuits/line width}}
        \pgfpathmoveto{\pgfpointadd{\plus}{\pgfpoint{0cm}{0.5 * \height}}}
        \pgfpathlineto{\pgfpointadd{\plus}{\pgfpoint{0cm}{-0.5 * \height}}}
        %\pgfusepath{stroke}
        \pgfpathmoveto{\pgfpointadd{\minus}{\pgfpoint{0cm}{0.5 * \height}}}
        \pgfpathlineto{\pgfpointadd{\minus}{\pgfpoint{0cm}{-0.5 * \height}}}
        \pgfusepath{stroke}
    }
}
\makeatother

% vim: ft=plaintex

\tikzset{
    circuits/resistor/width/.initial = 0.6cm,
    circuits/resistor/height/.initial = 0.25cm,
    circuits/resistor/segments/.initial = 3,
    circuits/resistor/terminal extension/.initial = 0.05cm,
    resistor/.style = {americanresistorshape}
}

\makeatletter
\pgfdeclareshape{americanresistorshape}
{
    \saveddimen{\width}{\pgf@x=\pgfkeysvalueof{/tikz/circuits/resistor/width}}
    \saveddimen{\height}{\pgf@x=\pgfkeysvalueof{/tikz/circuits/resistor/height}}
    \saveddimen{\extension}{\pgf@x=\pgfkeysvalueof{/tikz/circuits/resistor/terminal extension}}
    \savedmacro{\segments}{\renewcommand{\segments}[0]{\pgfkeysvalueof{/tikz/circuits/resistor/segments}}}
    \savedanchor{\leftpoint}{
        \pgfpointadd{\pgfpointorigin}{\pgfpoint{-0.5 * \pgfkeysvalueof{/tikz/circuits/resistor/width}- \pgfkeysvalueof{/tikz/circuits/resistor/terminal extension}}{0cm}}
    }
    \savedanchor{\rightpoint}{
        \pgfpointadd{\pgfpointorigin}{\pgfpoint{0.5 * \pgfkeysvalueof{/tikz/circuits/resistor/width} + \pgfkeysvalueof{/tikz/circuits/resistor/terminal extension}}{0cm}}
    }

    % electrical terminals
    \anchor{plus}{\leftpoint}
    \anchor{minus}{\rightpoint}
    % general anchors
    \anchor{center}{\pgfpointorigin}
    \anchor{west}{\leftpoint}
    \anchor{east}{\rightpoint}
    \anchor{north}{\pgfpointadd{\pgfpointorigin}{\pgfpoint{0cm}{0.5 * \height}}}
    \anchor{south}{\pgfpointadd{\pgfpointorigin}{\pgfpoint{0cm}{-0.5 * \height}}}
    \anchor{south east}{\pgfpointadd{\pgfpointorigin}{\pgfpoint{0.5 * \height}{-0.5 * \width}}}
    \anchor{south west}{\pgfpointadd{\pgfpointorigin}{\pgfpoint{-0.5 * \height}{-0.5 * \width}}}
    \anchor{north east}{\pgfpointadd{\pgfpointorigin}{\pgfpoint{0.5 * \height}{0.5 * \width}}}
    \anchor{north west}{\pgfpointadd{\pgfpointorigin}{\pgfpoint{-0.5 * \height}{0.5 * \width}}}
    \beforebackgroundpath{
        \pgfsetlinewidth{\pgfkeysvalueof{/tikz/circuits/line width}}
        \pgfsetmiterjoin
        \pgfsetmiterlimit{5}
        \pgfpathmoveto{\leftpoint}
        \pgfpathlineto{\pgfpointadd{\leftpoint}{\pgfpoint{\extension}{0cm}}}
        \foreach \x [count=\xi, evaluate=\xi as \y using isodd(\xi)-0.5] in {0.25,0.75,...,\segments}
        {
            \pgfpathlineto{\pgfpointadd{\leftpoint}{\pgfpoint{\extension + 4 * \x * \width / (4 * \segments)}{\y * \height}}}
        }
        \pgfpathlineto{\pgfpointadd{\rightpoint}{\pgfpoint{-\extension}{0cm}}}
        \pgfpathlineto{\rightpoint}
        \pgfusepath{stroke}
    }
}
\pgfdeclareshape{europeenresistorshape}
{
    \saveddimen{\width}{\pgf@x=\pgfkeysvalueof{/tikz/circuits/resistor/width}}
    \saveddimen{\height}{\pgf@x=\pgfkeysvalueof{/tikz/circuits/resistor/height}}
    \savedanchor{\lowerleft}{
        \pgfpointadd{\pgfpointorigin}{\pgfpoint{-0.5 * \pgfkeysvalueof{/tikz/circuits/resistor/width}}{-0.5 * \pgfkeysvalueof{/tikz/circuits/resistor/height}}}
    }
    \savedanchor{\upperright}{
        \pgfpointadd{\pgfpointorigin}{\pgfpoint{0.5 * \pgfkeysvalueof{/tikz/circuits/resistor/width}}{0.5 * \pgfkeysvalueof{/tikz/circuits/resistor/height}}}
    }

    % electrical terminals
    \anchor{plus}{\pgfpointadd{\pgfpointorigin}{\pgfpoint{-0.5 * \width}{0cm}}}
    \anchor{minus}{\pgfpointadd{\pgfpointorigin}{\pgfpoint{0.5 * \width}{0cm}}}
    % general anchors
    \anchor{lowerleft}{\lowerleft}
    \anchor{upperright}{\upperright}
    \anchor{center}{\pgfpointorigin}
    \anchor{west}{\pgfpointadd{\pgfpointorigin}{\pgfpoint{-0.5 * \width}{0cm}}}
    \anchor{east}{\pgfpointadd{\pgfpointorigin}{\pgfpoint{0.5 * \width}{0cm}}}
    \anchor{north}{\pgfpointadd{\pgfpointorigin}{\pgfpoint{0cm}{0.5 * \height}}}
    \anchor{south}{\pgfpointadd{\pgfpointorigin}{\pgfpoint{0cm}{-0.5 * \height}}}
    \anchor{south east}{\pgfpointadd{\pgfpointorigin}{\pgfpoint{0.5 * \height}{-0.5 * \width}}}
    \anchor{south west}{\pgfpointadd{\pgfpointorigin}{\pgfpoint{-0.5 * \height}{-0.5 * \width}}}
    \anchor{north east}{\pgfpointadd{\pgfpointorigin}{\pgfpoint{0.5 * \height}{0.5 * \width}}}
    \anchor{north west}{\pgfpointadd{\pgfpointorigin}{\pgfpoint{-0.5 * \height}{0.5 * \width}}}
    \beforebackgroundpath{
        \pgfsetlinewidth{\pgfkeysvalueof{/tikz/circuits/line width}}
        \pgfpathrectanglecorners{\lowerleft}{\upperright}
        \pgfusepath{stroke}
    }
}
\makeatother

% vim: ft=plaintex

\tikzset{
    circuits/inductor/width/.initial = 0.8cm,
    circuits/inductor/height/.initial = 0.266667cm,
    circuits/inductor/segments/.initial = 3,
    circuits/inductor/terminal extension/.initial = 0.05cm,
    inductor/.style = {americaninductorshape}
}

\makeatletter
\pgfdeclareshape{americaninductorshape}
{
    \saveddimen{\width}{\pgf@x=\pgfkeysvalueof{/tikz/circuits/inductor/width}}
    \saveddimen{\height}{\pgf@x=\pgfkeysifdefined{/tikz/circuits/inductor/height}{\pgfkeysvalueof{/tikz/circuits/inductor/height}}{\pgfkeysvalueof{/tikz/circuits/inductor/width}}}
    \saveddimen{\extension}{\pgf@x=\pgfkeysvalueof{/tikz/circuits/inductor/terminal extension}}
    \savedmacro{\segments}{\renewcommand{\segments}[0]{\pgfkeysvalueof{/tikz/circuits/inductor/segments}}}
    \savedanchor{\leftpoint}{
        \pgfpointadd{\pgfpointorigin}{\pgfpoint{-0.5 * \pgfkeysvalueof{/tikz/circuits/inductor/width}- \pgfkeysvalueof{/tikz/circuits/inductor/terminal extension}}{0cm}}
    }
    \savedanchor{\rightpoint}{
        \pgfpointadd{\pgfpointorigin}{\pgfpoint{0.5 * \pgfkeysvalueof{/tikz/circuits/inductor/width} + \pgfkeysvalueof{/tikz/circuits/inductor/terminal extension}}{0cm}}
    }

    % electrical terminals
    \anchor{plus}{\leftpoint}
    \anchor{minus}{\rightpoint}
    % general anchors
    \anchor{center}{\pgfpointorigin}
    \anchor{west}{\leftpoint}
    \anchor{inner west}{
        \pgfpointadd{\pgfpointorigin}{
            \pgfpoint
            {-0.5 * \width}
            {0cm}
        }
    }
    \anchor{east}{\rightpoint}
    \anchor{north}{\pgfpointadd{\pgfpointorigin}{\pgfpoint{0cm}{0.5 * \height}}}
    \anchor{south}{\pgfpointadd{\pgfpointorigin}{\pgfpoint{0cm}{-0.5 * \height}}}
    \anchor{south east}{\pgfpointadd{\pgfpointorigin}{\pgfpoint{0.5 * \height}{-0.5 * \width}}}
    \anchor{south west}{\pgfpointadd{\pgfpointorigin}{\pgfpoint{-0.5 * \height}{-0.5 * \width}}}
    \anchor{north east}{\pgfpointadd{\pgfpointorigin}{\pgfpoint{0.5 * \height}{0.5 * \width}}}
    \anchor{north west}{\pgfpointadd{\pgfpointorigin}{\pgfpoint{-0.5 * \height}{0.5 * \width}}}
    \beforebackgroundpath{
        \pgfsetlinewidth{\pgfkeysvalueof{/tikz/circuits/line width}}
        \pgfpathmoveto{\leftpoint}
        \pgfpathlineto{\pgfpointadd{\leftpoint}{\pgfpoint{\extension}{0cm}}}
        \foreach \x in {1,...,\segments}
        {
            \pgfpatharc{180}{0}{\width / \segments / 2 and \height/2}
        }
        \pgfpathlineto{\pgfpointadd{\rightpoint}{\pgfpoint{-\extension}{0cm}}}
        \pgfpathlineto{\rightpoint}
        \pgfusepath{stroke}
    }
}
\pgfdeclareshape{europeeninductorshape}
{
    \saveddimen{\width}{\pgf@x=\pgfkeysvalueof{/tikz/circuits/inductor/width}}
    \saveddimen{\height}{\pgf@x=\pgfkeysvalueof{/tikz/circuits/inductor/height}}
    \savedanchor{\lowerleft}{
        \pgfpointadd{\pgfpointorigin}{\pgfpoint{-0.5 * \pgfkeysvalueof{/tikz/circuits/inductor/width}}{-0.5 * \pgfkeysvalueof{/tikz/circuits/inductor/height}}}
    }
    \savedanchor{\upperright}{
        \pgfpointadd{\pgfpointorigin}{\pgfpoint{0.5 * \pgfkeysvalueof{/tikz/circuits/inductor/width}}{0.5 * \pgfkeysvalueof{/tikz/circuits/inductor/height}}}
    }

    % electrical terminals
    \anchor{plus}{\pgfpointadd{\pgfpointorigin}{\pgfpoint{-0.5 * \width}{0cm}}}
    \anchor{minus}{\pgfpointadd{\pgfpointorigin}{\pgfpoint{0.5 * \width}{0cm}}}
    % general anchors
    \anchor{lowerleft}{\lowerleft}
    \anchor{upperright}{\upperright}
    \anchor{center}{\pgfpointorigin}
    \anchor{west}{\pgfpointadd{\pgfpointorigin}{\pgfpoint{-0.5 * \width}{0cm}}}
    \anchor{east}{\pgfpointadd{\pgfpointorigin}{\pgfpoint{0.5 * \width}{0cm}}}
    \anchor{north}{\pgfpointadd{\pgfpointorigin}{\pgfpoint{0cm}{0.5 * \height}}}
    \anchor{south}{\pgfpointadd{\pgfpointorigin}{\pgfpoint{0cm}{-0.5 * \height}}}
    \anchor{south east}{\pgfpointadd{\pgfpointorigin}{\pgfpoint{0.5 * \height}{-0.5 * \width}}}
    \anchor{south west}{\pgfpointadd{\pgfpointorigin}{\pgfpoint{-0.5 * \height}{-0.5 * \width}}}
    \anchor{north east}{\pgfpointadd{\pgfpointorigin}{\pgfpoint{0.5 * \height}{0.5 * \width}}}
    \anchor{north west}{\pgfpointadd{\pgfpointorigin}{\pgfpoint{-0.5 * \height}{0.5 * \width}}}
    \beforebackgroundpath{
        \pgfsetlinewidth{\pgfkeysvalueof{/tikz/circuits/line width}}
        \pgfpathrectanglecorners{\lowerleft}{\upperright}
        \pgfusepath{fill}
    }
}
\makeatother

% vim: ft=plaintex

\newif\ifmosfetdrawbulk
\newif\ifmosfetdrawbulkarrow
\newif\ifmosfetmirror
\newif\ifmosfetnoarrow
\newif\ifmosfetfilldot

\tikzset{
    circuits/mosfet/scale/.initial=1, % not implemented
    circuits/mosfet/gate width/.initial={0.6cm},
    circuits/mosfet/gate skip/.initial={0.1cm},
    circuits/mosfet/drain length/.initial={0.2cm},
    circuits/mosfet/source length/.initial={0.2cm},
    circuits/mosfet/channel height/.initial={0.32cm},
    circuits/mosfet/channel length/.initial={0.6cm},
    circuits/mosfet/arrow length/.initial={1.75mm},
    circuits/mosfet/arrow width/.initial={1.75mm},
    circuits/mosfet/gate line width/.initial={1.2pt},
    circuits/mosfet/bulk line width/.initial={0.8pt},
    circuits/mosfet/gate dot radius/.initial=1.5pt,
    circuits/mosfet/gate dot line width/.initial=0.8pt,
    % mosfet options
    circuits/mosfet/no source arrow/.is if=mosfetnoarrow,
    circuits/mosfet/fill gate dot/.is if=mosfetfilldot,
    circuits/mosfet/draw bulk/.is if=mosfetdrawbulk,
    circuits/mosfet/draw bulk arrow/.is if=mosfetdrawbulkarrow,
    circuits/mosfet/mirror/.is if=mosfetmirror,
}


\makeatletter
\pgfdeclareshape{nmos}
{
    \saveddimen{\gatewidth}{\pgf@x=\pgfkeysvalueof{/tikz/circuits/mosfet/gate width}}
    \saveddimen{\gateskip}{\pgf@x=\pgfkeysvalueof{/tikz/circuits/mosfet/gate skip}}
    \saveddimen{\drainlength}{\pgf@x=\pgfkeysvalueof{/tikz/circuits/mosfet/drain length}}
    \saveddimen{\sourcelength}{\pgf@x=\pgfkeysvalueof{/tikz/circuits/mosfet/source length}}
    \saveddimen{\channellength}{\pgf@x=\pgfkeysvalueof{/tikz/circuits/mosfet/channel length}}
    \saveddimen{\channelheight}{\pgf@x=\pgfkeysvalueof{/tikz/circuits/mosfet/channel height}}
    \saveddimen{\arrowlength}{\pgf@x=\pgfkeysvalueof{/tikz/circuits/mosfet/arrow length}}
    \saveddimen{\arrowwidth}{\pgf@x=\pgfkeysvalueof{/tikz/circuits/mosfet/arrow width}}

    % electrical terminals
    \anchor{drain}{\pgfpointadd{\pgfpointorigin}{\pgfpoint{0cm}{0.5 * \channellength + \drainlength}}}
    \anchor{source}{\pgfpointadd{\pgfpointorigin}{\pgfpoint{0cm}{-0.5 * \channellength - \sourcelength}}}
    \anchor{gate}{\pgfpointadd{\pgfpointorigin}{\pgfpoint{-\channelheight -\gateskip}{0cm}}}
    \anchor{bulk}{\pgfpointorigin}
    % general anchors
    \anchor{center}{\pgfpointadd{\pgfpointorigin}{\pgfpoint{(-\channelheight - \gateskip)/2}{0cm}}}
    \anchor{north}{\pgfpointadd{\pgfpointorigin}{\pgfpoint{(-\channelheight - \gateskip)/2}{0.5 * \channellength + \drainlength}}}
    \anchor{south}{\pgfpointadd{\pgfpointorigin}{\pgfpoint{(-\channelheight - \gateskip)/2}{-0.5 * \channellength - \drainlength}}}
    \anchor{west}{\pgfpointadd{\pgfpointorigin}{\pgfpoint{-\channelheight -\gateskip}{0cm}}}
    \anchor{east}{\pgfpointorigin}
    \anchor{north west}{\pgfpointadd{\pgfpointorigin}{\pgfpoint{-\channelheight - \gateskip}{0.5 * \channellength + \drainlength}}}
    \anchor{north east}{\pgfpointadd{\pgfpointorigin}{\pgfpoint{0cm}{0.5 * \channellength + \drainlength}}}
    \anchor{south west}{\pgfpointadd{\pgfpointorigin}{\pgfpoint{-\channelheight - \gateskip}{-0.5 * \channellength - \drainlength}}}
    \anchor{south east}{\pgfpointadd{\pgfpointorigin}{\pgfpoint{0cm}{-0.5 * \channellength - \drainlength}}}
    \beforebackgroundpath{
        \pgfsetbuttcap
        \pgfsetlinewidth{\pgfkeysvalueof{/tikz/circuits/line width}}
        %% draw channel, source and drain extensions
        % move to drain
        \pgfpathmoveto{\pgfpointadd{\pgfpointorigin}{\pgfpoint{0cm}{0.5 * \channellength + \drainlength}}}
        % draw drain extension
        \pgfpathlineto{\pgfpointadd{\pgfpointorigin}{\pgfpoint{0cm}{0.5 * \channellength}}}
        % draw to channel
        \pgfpathlineto{\pgfpointadd{\pgfpointorigin}{\pgfpoint{-\channelheight}{0.5 * \channellength}}}
        % draw half of channel
        \pgfpathlineto{\pgfpointadd{\pgfpointorigin}{\pgfpoint{-\channelheight}{0cm}}}
        % draw other half of channel (starting from source
        \pgfpathlineto{\pgfpointadd{\pgfpointorigin}{\pgfpoint{-\channelheight}{-0.5 * \channellength}}}
        % move to 'inner' source (without extension)
        \pgfpathlineto{\pgfpointadd{\pgfpointorigin}{\pgfpoint{0cm}{-0.5 * \channellength}}}
        % draw source extension
        \pgfpathlineto{\pgfpointadd{\pgfpointorigin}{\pgfpoint{0cm}{-0.5 * \channellength - \sourcelength}}}
        \pgfusepath{stroke}
        %% draw gate
        \pgfsetlinewidth{\pgfkeysvalueof{/tikz/circuits/mosfet/gate line width}}
        \pgfpathmoveto{\pgfpointadd{\pgfpointorigin}{\pgfpoint{-\channelheight - \gateskip}{0.5 * \channellength + 0.5 * \pgfkeysvalueof{/tikz/circuits/line width}}}}
        \pgfpathlineto{\pgfpointadd{\pgfpointorigin}{\pgfpoint{-\channelheight - \gateskip}{-0.5 * \channellength - 0.5 * \pgfkeysvalueof{/tikz/circuits/line width}}}}
        \pgfusepath{stroke}
        % draw bulk, source arrow and gate circle, depending on the display options
        \ifmosfetdrawbulk
            \ifmosfetnoarrow
                % draw bulk (with arrow)
                \pgfsetlinewidth{\pgfkeysvalueof{/tikz/circuits/mosfet/bulk line width}}
                \pgfpathmoveto{\pgfpointorigin}
                \pgfpathlineto{\pgfpointadd{\pgfpointorigin}{\pgfpoint{-\channelheight}{0cm}}}
                \pgfsetarrows{-{Stealth[length=\pgfkeysvalueof{/tikz/circuits/mosfet/arrow length}, width=\pgfkeysvalueof{/tikz/circuits/mosfet/arrow width}]}}
                \pgfusepath{stroke}
            \else
                % draw bulk (without arrow)
                \pgfsetlinewidth{\pgfkeysvalueof{/tikz/circuits/mosfet/bulk line width}}
                \pgfpathmoveto{\pgfpointorigin}
                \pgfpathlineto{\pgfpointadd{\pgfpointorigin}{\pgfpoint{-\channelheight}{0cm}}}
                \pgfusepath{stroke}
                % draw source arrow
                \pgfsetlinewidth{0.8*\pgfkeysvalueof{/tikz/circuits/line width}}
                \pgfpathmoveto{\pgfpointadd{\pgfpointorigin}{\pgfpoint{-\channelheight}{-0.5 * \channellength}}}
                \pgfpathlineto{\pgfpointadd{\pgfpointorigin}{\pgfpoint{0cm}{-0.5 * \channellength}}}
                \pgfsetarrows{-{Stealth[length=\pgfkeysvalueof{/tikz/circuits/mosfet/arrow length}, width=\pgfkeysvalueof{/tikz/circuits/mosfet/arrow width}]}}
                \pgfusepath{stroke}
                \pgfsetarrowsend{}
            \fi
        \else
            \ifmosfetnoarrow
                % no circle in nmos
            \else
                % draw source arrow
                \pgfsetlinewidth{0.8*\pgfkeysvalueof{/tikz/circuits/line width}}
                \pgfpathmoveto{\pgfpointadd{\pgfpointorigin}{\pgfpoint{-\channelheight}{-0.5 * \channellength}}}
                \pgfpathlineto{\pgfpointadd{\pgfpointorigin}{\pgfpoint{0cm}{-0.5 * \channellength}}}
                \pgfsetarrows{-{Stealth[length=\pgfkeysvalueof{/tikz/circuits/mosfet/arrow length}, width=\pgfkeysvalueof{/tikz/circuits/mosfet/arrow width}]}}
                \pgfusepath{stroke}
                \pgfsetarrowsend{}
            \fi
        \fi
    }
}
\makeatother

% vim: ft=plaintex nowrap

\newif\ifmosfetdrawbulk
\newif\ifmosfetdrawbulkarrow
\newif\ifmosfetmirror
\newif\ifmosfetnoarrow
\newif\ifmosfetfilldot

\tikzset{
    circuits/mosfet/scale/.initial=1, % not implemented
    circuits/mosfet/gate width/.initial={0.6cm},
    circuits/mosfet/gate skip/.initial={0.1cm},
    circuits/mosfet/drain length/.initial={0.2cm},
    circuits/mosfet/source length/.initial={0.2cm},
    circuits/mosfet/channel height/.initial={0.32cm},
    circuits/mosfet/channel length/.initial={0.6cm},
    circuits/mosfet/arrow length/.initial={1.75mm},
    circuits/mosfet/arrow width/.initial={1.75mm},
    circuits/mosfet/gate line width/.initial={1.2pt},
    circuits/mosfet/bulk line width/.initial={0.8pt},
    circuits/mosfet/gate dot radius/.initial=1.5pt,
    circuits/mosfet/gate dot line width/.initial=0.8pt,
    % mosfet options
    circuits/mosfet/no source arrow/.is if=mosfetnoarrow,
    circuits/mosfet/fill gate dot/.is if=mosfetfilldot,
    circuits/mosfet/draw bulk/.is if=mosfetdrawbulk,
    circuits/mosfet/draw bulk arrow/.is if=mosfetdrawbulkarrow,
    circuits/mosfet/mirror/.is if=mosfetmirror,
}


\newif\ifpmosfillgatecircle

\tikzset{
    circuits/pmos/fill gate circle/.is if=pmosfillgatecircle
}

\makeatletter
\pgfdeclareshape{pmos}
{
    \saveddimen{\gatewidth}{\pgf@x=\pgfkeysvalueof{/tikz/circuits/mosfet/gate width}}
    \saveddimen{\gateskip}{\pgf@x=\pgfkeysvalueof{/tikz/circuits/mosfet/gate skip}}
    \saveddimen{\drainlength}{\pgf@x=\pgfkeysvalueof{/tikz/circuits/mosfet/drain length}}
    \saveddimen{\sourcelength}{\pgf@x=\pgfkeysvalueof{/tikz/circuits/mosfet/source length}}
    \saveddimen{\channellength}{\pgf@x=\pgfkeysvalueof{/tikz/circuits/mosfet/channel length}}
    \saveddimen{\channelheight}{\pgf@x=\pgfkeysvalueof{/tikz/circuits/mosfet/channel height}}
    \saveddimen{\arrowlength}{\pgf@x=\pgfkeysvalueof{/tikz/circuits/mosfet/arrow length}}
    \saveddimen{\arrowwidth}{\pgf@x=\pgfkeysvalueof{/tikz/circuits/mosfet/arrow width}}

    % electrical terminals
    \anchor{drain}{\pgfpointadd{\pgfpointorigin}{\pgfpoint{0cm}{-0.5 * \channellength - \drainlength}}}
    \anchor{source}{\pgfpointadd{\pgfpointorigin}{\pgfpoint{0cm}{0.5 * \channellength + \sourcelength}}}
    \anchor{gate}{\pgfpointadd{\pgfpointorigin}{\pgfpoint{-\channelheight -\gateskip}{0cm}}}
    \anchor{bulk}{\pgfpointorigin}
    % general anchors
    \anchor{center}{\pgfpointadd{\pgfpointorigin}{\pgfpoint{(-\channelheight - \gateskip)/2}{0cm}}}
    \anchor{north}{\pgfpointadd{\pgfpointorigin}{\pgfpoint{(-\channelheight - \gateskip)/2}{0.5 * \channellength + \drainlength}}}
    \anchor{south}{\pgfpointadd{\pgfpointorigin}{\pgfpoint{(-\channelheight - \gateskip)/2}{-0.5 * \channellength - \drainlength}}}
    \anchor{west}{\pgfpointadd{\pgfpointorigin}{\pgfpoint{-\channelheight -\gateskip}{0cm}}}
    \anchor{east}{\pgfpointorigin}
    \anchor{north west}{\pgfpointadd{\pgfpointorigin}{\pgfpoint{-\channelheight - \gateskip}{0.5 * \channellength + \drainlength}}}
    \anchor{north east}{\pgfpointadd{\pgfpointorigin}{\pgfpoint{0cm}{0.5 * \channellength + \drainlength}}}
    \anchor{south west}{\pgfpointadd{\pgfpointorigin}{\pgfpoint{-\channelheight - \gateskip}{-0.5 * \channellength - \drainlength}}}
    \anchor{south east}{\pgfpointadd{\pgfpointorigin}{\pgfpoint{0cm}{-0.5 * \channellength - \drainlength}}}
    \beforebackgroundpath{
        \pgfsetbuttcap
        \pgfsetlinewidth{\pgfkeysvalueof{/tikz/circuits/line width}}
        %% draw channel, source and drain extensions
        % move to source
        \pgfpathmoveto{\pgfpointadd{\pgfpointorigin}{\pgfpoint{0cm}{0.5 * \channellength + \sourcelength}}}
        % draw drain extension
        \pgfpathlineto{\pgfpointadd{\pgfpointorigin}{\pgfpoint{0cm}{0.5 * \channellength}}}
        % draw to channel
        \pgfpathlineto{\pgfpointadd{\pgfpointorigin}{\pgfpoint{-\channelheight}{0.5 * \channellength}}}
        % draw half of channel
        \pgfpathlineto{\pgfpointadd{\pgfpointorigin}{\pgfpoint{-\channelheight}{0cm}}}
        % draw other half of channel (starting from drain
        \pgfpathlineto{\pgfpointadd{\pgfpointorigin}{\pgfpoint{-\channelheight}{-0.5 * \channellength}}}
        % move to 'inner' source (without extension)
        \pgfpathlineto{\pgfpointadd{\pgfpointorigin}{\pgfpoint{0cm}{-0.5 * \channellength}}}
        % draw drain extension
        \pgfpathlineto{\pgfpointadd{\pgfpointorigin}{\pgfpoint{0cm}{-0.5 * \channellength - \drainlength}}}
        \pgfusepath{stroke}
        %% draw gate
        \pgfsetlinewidth{\pgfkeysvalueof{/tikz/circuits/mosfet/gate line width}}
        \pgfpathmoveto{\pgfpointadd{\pgfpointorigin}{\pgfpoint{-\channelheight - \gateskip}{0.5 * \channellength + 0.5 * \pgfkeysvalueof{/tikz/circuits/line width}}}}
        \pgfpathlineto{\pgfpointadd{\pgfpointorigin}{\pgfpoint{-\channelheight - \gateskip}{-0.5 * \channellength - 0.5 * \pgfkeysvalueof{/tikz/circuits/line width}}}}
        \pgfusepath{stroke}
        % draw bulk, source arrow and gate circle, depending on the display options
        \ifmosfetdrawbulk
            \ifmosfetnoarrow
                % draw bulk (with arrow)
                \pgfsetlinewidth{\pgfkeysvalueof{/tikz/circuits/mosfet/bulk line width}}
                \pgfpathmoveto{\pgfpointadd{\pgfpointorigin}{\pgfpoint{-\channelheight}{0cm}}}
                \pgfpathlineto{\pgfpointorigin}
                \pgfsetarrows{-{Stealth[length=\pgfkeysvalueof{/tikz/circuits/mosfet/arrow length}, width=\pgfkeysvalueof{/tikz/circuits/mosfet/arrow width}]}}
                \pgfusepath{stroke}
            \else
                % draw bulk (without arrow)
                \pgfsetlinewidth{\pgfkeysvalueof{/tikz/circuits/mosfet/bulk line width}}
                \pgfpathmoveto{\pgfpointorigin}
                \pgfpathlineto{\pgfpointadd{\pgfpointorigin}{\pgfpoint{-\channelheight}{0cm}}}
                \pgfusepath{stroke}
                % draw source arrow
                \pgfsetlinewidth{0.8*\pgfkeysvalueof{/tikz/circuits/line width}}
                \pgfpathmoveto{\pgfpointadd{\pgfpointorigin}{\pgfpoint{0cm}{0.5 * \channellength}}}
                \pgfpathlineto{\pgfpointadd{\pgfpointorigin}{\pgfpoint{-\channelheight}{0.5 * \channellength}}}
                \pgfsetarrows{-{Stealth[length=\pgfkeysvalueof{/tikz/circuits/mosfet/arrow length}, width=\pgfkeysvalueof{/tikz/circuits/mosfet/arrow width}]}}
                \pgfusepath{stroke}
                \pgfsetarrowsend{}
            \fi
        \else
            \ifmosfetnoarrow
                \pgfpathcircle{\pgfpointadd{\pgfpointorigin}{\pgfpoint{-\channelheight - \gateskip - \pgfkeysvalueof{/tikz/circuits/mosfet/gate dot radius}}{0cm}}}{\pgfkeysvalueof{/tikz/circuits/mosfet/gate dot radius}}
                \ifpmosfillgatecircle
                    \pgfusepath{stroke, fill}
                \else
                    \pgfusepath{stroke}
                \fi
            \else
                \pgfsetlinewidth{0.8*\pgfkeysvalueof{/tikz/circuits/line width}}
                \pgfpathmoveto{\pgfpointadd{\pgfpointorigin}{\pgfpoint{0cm}{0.5 * \channellength}}}
                \pgfpathlineto{\pgfpointadd{\pgfpointorigin}{\pgfpoint{-\channelheight}{0.5 * \channellength}}}
                \pgfsetarrows{-{Stealth[length=\pgfkeysvalueof{/tikz/circuits/mosfet/arrow length}, width=\pgfkeysvalueof{/tikz/circuits/mosfet/arrow width}]}}
                \pgfusepath{stroke}
                \pgfsetarrowsend{}
            \fi
        \fi
    }
}
\makeatother

% vim: ft=plaintex nowrap

\tikzset{
    circuits/currentsource/radius/.initial = 0.25cm,
    circuits/currentsource/extension factor/.initial = 1.0,
    circuits/currentsource/terminal extension/.initial = 0cm,
    circuits/currentsource/arrow length/.initial={1.75mm},
    circuits/currentsource/arrow width/.initial={1.75mm},
    currentsource/.style = {currentsourceshape}
}

\makeatletter
\pgfdeclareshape{currentsourceshape}
{
    \saveddimen{\radius}{\pgf@x=\pgfkeysvalueof{/tikz/circuits/currentsource/radius}}
    \saveddimen{\extension}{\pgf@x=\pgfkeysvalueof{/tikz/circuits/currentsource/terminal extension}}
    \savedmacro{\extensionfactor}{\edef\extensionfactor{\pgfkeysvalueof{/tikz/circuits/currentsource/extension factor}}}

    % general anchors
    \anchor{center}{\pgfpointorigin}
    \anchor{west}{\pgfpointadd{\pgfpointorigin}{\pgfpoint{-\radius}{0cm}}}
    \anchor{east}{\pgfpointadd{\pgfpointorigin}{\pgfpoint{\radius}{0cm}}}
    \anchor{outer north}{\pgfpointadd{\pgfpointorigin}{\pgfpoint{0cm}{ \radius * (1 + \extensionfactor)}}}
    \anchor{outer south}{\pgfpointadd{\pgfpointorigin}{\pgfpoint{0cm}{-\radius * (1 + \extensionfactor)}}}
    \anchor{north}{\pgfpointadd{\pgfpointorigin}{\pgfpoint{0cm}{ \radius + \extension}}}
    \anchor{south}{\pgfpointadd{\pgfpointorigin}{\pgfpoint{0cm}{-\radius - \extension}}}
    \anchor{south east}{\pgfpointadd{\pgfpointorigin}{\pgfpoint{-\radius - \extension}{-\radius}}}
    \anchor{south west}{\pgfpointadd{\pgfpointorigin}{\pgfpoint{-\radius - \extension}{-\radius}}}
    \anchor{north east}{\pgfpointadd{\pgfpointorigin}{\pgfpoint{ \radius + \extension}{ \radius}}}
    \anchor{north west}{\pgfpointadd{\pgfpointorigin}{\pgfpoint{ \radius + \extension}{ \radius}}}
    \beforebackgroundpath{
        \pgfsetlinewidth{\pgfkeysvalueof{/tikz/circuits/line width}}
        \pgfpathcircle{\pgfpointorigin}{\radius}
        \pgfusepath{stroke}
        % draw terminals
        \pgfpathmoveto{\pgfpointadd{\pgfpointorigin}{\pgfpoint{0cm}{-\radius - \extension}}}
        \pgfpathlineto{\pgfpointadd{\pgfpointorigin}{\pgfpoint{0cm}{-\radius}}}
        \pgfpathmoveto{\pgfpointadd{\pgfpointorigin}{\pgfpoint{0cm}{ \radius + \extension}}}
        \pgfpathlineto{\pgfpointadd{\pgfpointorigin}{\pgfpoint{0cm}{ \radius}}}
        \pgfusepath{stroke}
        % draw arrow
        \pgfpathmoveto{\pgfpointadd{\pgfpointorigin}{\pgfpoint{0cm}{0.7 * \radius}}}
        \pgfpathlineto{\pgfpointadd{\pgfpointorigin}{\pgfpoint{0cm}{-0.7 * \radius}}}
        \pgfsetarrows{-{Stealth[length=\pgfkeysvalueof{/tikz/circuits/currentsource/arrow length}, width=\pgfkeysvalueof{/tikz/circuits/currentsource/arrow width}]}}
        \pgfusepath{stroke}
    }
}
\makeatother

% vim: ft=plaintex

\tikzset{
    circuits/ground/lines/.initial = 3,
    circuits/ground/width/.initial = 0.15cm,
    circuits/ground/height/.initial = 0.075cm,
    circuits/ground/last width factor/.initial = 0.4,
    ground/.style = {groundshape}
}

\makeatletter
\pgfdeclareshape{groundshape}
{
    \savedmacro{\lines}{\edef\lines{\pgfkeysvalueof{/tikz/circuits/ground/lines}}}
    \savedmacro{\lastwidthfactor}{\edef\lastwidthfactor{\pgfkeysvalueof{/tikz/circuits/ground/last width factor}}}
    \saveddimen{\width}{\pgf@x=\pgfkeysvalueof{/tikz/circuits/ground/width}}
    \saveddimen{\height}{\pgf@x=\pgfkeysvalueof{/tikz/circuits/ground/height}}
    % general anchors
    \anchor{center}{\pgfpointorigin}
    \anchor{north west}{\pgfpointadd{\pgfpointorigin}{\pgfpoint{-0.5 * \width}{0.5 * \height}}}
    \anchor{west}{\pgfpointadd{\pgfpointorigin}{\pgfpoint{-0.5 * \width}{0cm}}}
    \anchor{east}{\pgfpointadd{\pgfpointorigin}{\pgfpoint{ 0.5 * \width}{0cm}}}
    \anchor{north}{\pgfpointadd{\pgfpointorigin}{\pgfpoint{0cm}{ 0.5 * \height}}}
    \anchor{south}{\pgfpointadd{\pgfpointorigin}{\pgfpoint{0cm}{-0.5 * \height}}}
    \beforebackgroundpath{
        \pgfsetlinewidth{\pgfkeysvalueof{/tikz/circuits/line width}}
        \foreach \y in {1,2,...,\lines}
        {
            \pgfpathmoveto{\pgfpointadd{\pgfpointorigin}{\pgfpoint{ (1 - (1 - \lastwidthfactor) / (\lines - 1) * (\y - 1)) * \width}{0.5 * \height - \height / (\lines - 1) * (\y - 1)}}}
            \pgfpathlineto{\pgfpointadd{\pgfpointorigin}{\pgfpoint{-(1 - (1 - \lastwidthfactor) / (\lines - 1) * (\y - 1)) * \width}{0.5 * \height - \height / (\lines - 1) * (\y - 1)}}}
        }
        \pgfusepath{stroke}
    }
}
\makeatother

% vim: ft=plaintex nowrap

\tikzset{
    circuits/vdd/width/.initial = 0.25cm,
    vdd/.style = {vddshape}
}

\makeatletter
\pgfdeclareshape{vddshape}
{
    \saveddimen{\width}{\pgf@x=\pgfkeysvalueof{/tikz/circuits/vdd/width}}
    % general anchors
    \anchor{center}{\pgfpointorigin}
    \anchor{west}{\pgfpointadd{\pgfpointorigin}{\pgfpoint{-0.5 * \width}{0cm}}}
    \anchor{east}{\pgfpointadd{\pgfpointorigin}{\pgfpoint{ 0.5 * \width}{0cm}}}
    \anchor{north}{\pgfpointorigin}
    \anchor{south}{\pgfpointorigin}
    \beforebackgroundpath{
        \pgfsetlinewidth{\pgfkeysvalueof{/tikz/circuits/line width}}
            \pgfpathmoveto{\pgfpointadd{\pgfpointorigin}{\pgfpoint{-0.5 * \width}{0cm}}}
            \pgfpathlineto{\pgfpointadd{\pgfpointorigin}{\pgfpoint{0.5 * \width}{0cm}}}
        \pgfusepath{stroke}
    }
}
\makeatother

% vim: ft=plaintex nowrap

\tikzset{
    circuits/impedance/width/.initial = 0.6cm,
    circuits/impedance/height/.initial = 0.25cm,
    circuits/impedance/segments/.initial = 3,
    circuits/impedance/terminal extension/.initial = 0.05cm,
    impedance/.style = {impedanceshape}
}

\makeatletter
\pgfdeclareshape{impedanceshape}
{
    \saveddimen{\width}{\pgf@x=\pgfkeysvalueof{/tikz/circuits/impedance/width}}
    \saveddimen{\height}{\pgf@x=\pgfkeysvalueof{/tikz/circuits/impedance/height}}
    \savedanchor{\lowerleft}{
        \pgfpointadd{\pgfpointorigin}{\pgfpoint{-0.5 * \pgfkeysvalueof{/tikz/circuits/impedance/width}}{-0.5 * \pgfkeysvalueof{/tikz/circuits/impedance/height}}}
    }
    \savedanchor{\upperright}{
        \pgfpointadd{\pgfpointorigin}{\pgfpoint{0.5 * \pgfkeysvalueof{/tikz/circuits/impedance/width}}{0.5 * \pgfkeysvalueof{/tikz/circuits/impedance/height}}}
    }

    % electrical terminals
    \anchor{plus}{\pgfpointadd{\pgfpointorigin}{\pgfpoint{-0.5 * \width}{0cm}}}
    \anchor{minus}{\pgfpointadd{\pgfpointorigin}{\pgfpoint{0.5 * \width}{0cm}}}
    % general anchors
    \anchor{lowerleft}{\lowerleft}
    \anchor{upperright}{\upperright}
    \anchor{center}{\pgfpointorigin}
    \anchor{west}{\pgfpointadd{\pgfpointorigin}{\pgfpoint{-0.5 * \width}{0cm}}}
    \anchor{east}{\pgfpointadd{\pgfpointorigin}{\pgfpoint{0.5 * \width}{0cm}}}
    \anchor{north}{\pgfpointadd{\pgfpointorigin}{\pgfpoint{0cm}{0.5 * \height}}}
    \anchor{south}{\pgfpointadd{\pgfpointorigin}{\pgfpoint{0cm}{-0.5 * \height}}}
    \anchor{south east}{\pgfpointadd{\pgfpointorigin}{\pgfpoint{0.5 * \height}{-0.5 * \width}}}
    \anchor{south west}{\pgfpointadd{\pgfpointorigin}{\pgfpoint{-0.5 * \height}{-0.5 * \width}}}
    \anchor{north east}{\pgfpointadd{\pgfpointorigin}{\pgfpoint{0.5 * \height}{0.5 * \width}}}
    \anchor{north west}{\pgfpointadd{\pgfpointorigin}{\pgfpoint{-0.5 * \height}{0.5 * \width}}}
    \beforebackgroundpath{
        \pgfsetlinewidth{\pgfkeysvalueof{/tikz/circuits/line width}}
        \pgfpathrectanglecorners{\lowerleft}{\upperright}
        \pgfusepath{fill}
    }
}
\makeatother

% vim: ft=plaintex

\tikzset{
    circuits/amplifier/width/.initial = 1.25cm,
    circuits/amplifier/height/.initial = 1.25cm,
    amplifier/.style = {amplifiershape}
}

\makeatletter
\pgfdeclareshape{amplifiershape}
{
    \saveddimen{\width}{\pgf@x=\pgfkeysvalueof{/tikz/circuits/amplifier/width}}
    \saveddimen{\inputheight}{\pgf@x=\pgfkeysvalueof{/tikz/circuits/amplifier/height}}
    \savedanchor{\centerpoint}{\pgfpointorigin}
    \savedanchor{\inputplus}{%
        \pgfpointadd%
        {\pgfpointorigin}%
        {\pgfpoint%
            {-0.5 * \pgfkeysvalueof{/tikz/circuits/amplifier/width}}%
            {0.5 * \pgfkeysvalueof{/tikz/circuits/amplifier/height}}
        }%
    }
    \savedanchor{\inputminus}{%
        \pgfpointadd%
        {\pgfpointorigin}%
        {\pgfpoint%
            {-0.5 * \pgfkeysvalueof{/tikz/circuits/amplifier/width}}%
            {-0.5 * \pgfkeysvalueof{/tikz/circuits/amplifier/height}}
        }%
    }
    \savedanchor{\output}{%
        \pgfpointadd%
        {\pgfpointorigin}%
        {\pgfpoint%
            {0.5 * \pgfkeysvalueof{/tikz/circuits/amplifier/width}}%
            {0cm}
        }%
    }
    % electrical terminals (anchors)
    \anchor{in}{\pgfpointadd{\centerpoint}{\pgfpoint{-\width/2}{0cm}}}
    \anchor{out}{\output}
    \anchor{+power}{
        \pgfpointlineattime%
        {0.5}%
        {\pgfpointadd{\inputplus}{\pgfpoint{0cm}{0.5 * \inputheight}}}
        {\output}%
    }
    \anchor{-power}{
        \pgfpointlineattime%
        {0.5}%
        {\pgfpointadd{\inputminus}{\pgfpoint{0cm}{-0.5 * \inputheight}}}
        {\output}%
    }
    % regular anchors
    \anchor{center}{\centerpoint}
    \anchor{north}{
        \pgfpointadd{\inputplus}{\pgfpoint{\width/2}{0.5 * \inputheight}}
    }
    \anchor{south}{
        \pgfpointadd{\inputminus}{\pgfpoint{\width/2}{-0.5 * \inputheight}}
    }
    \anchor{east}{
        \pgfpointadd{\pgfpointorigin}{\pgfpoint{\width/2}{0cm}}
    }
    \anchor{west}{
        \pgfpointadd{\pgfpointorigin}{\pgfpoint{-\width/2}{0cm}}
    }
    \anchor{north west}{
        \pgfpointadd{\inputplus}{\pgfpoint{0cm}{0.5 * \inputheight}}
    }
    \anchor{south west}{
        \pgfpointadd{\inputminus}{\pgfpoint{0cm}{-0.5 * \inputheight}}
    }
    \anchor{north east}{
        \pgfpointadd{\inputplus}{\pgfpoint{\width}{0.5 * \inputheight}}
    }
    \anchor{south east}{
        \pgfpointadd{\inputminus}{\pgfpoint{\width}{-0.5 * \inputheight}}
    }
    \backgroundpath{
        \pgfsetlinewidth{\pgfkeysvalueof{/tikz/circuits/line width}}
        \pgfpathmoveto{\inputplus}
        \pgfpathlineto{\output}
        \pgfpathlineto{\inputminus}
        \pgfpathclose
        \pgfusepath{stroke}
    }
}
\makeatother

% vim: ft=plaintex nowrap

\tikzset{
    circuits/mixer/radius/.initial = 0.5cm,
    mixer/.style = {mixershape}
}

\makeatletter
\pgfdeclareshape{mixershape}
{
    \saveddimen{\radius}{\pgf@x=\pgfkeysvalueof{/tikz/circuits/mixer/radius}}

    % general anchors
    \anchor{center}{\pgfpointorigin}
    \anchor{west}{\pgfpointadd{\pgfpointorigin}{\pgfpoint{-\radius}{0cm}}}
    \anchor{east}{\pgfpointadd{\pgfpointorigin}{\pgfpoint{\radius}{0cm}}}
    \anchor{north}{\pgfpointadd{\pgfpointorigin}{\pgfpoint{0cm}{ \radius}}}
    \anchor{south}{\pgfpointadd{\pgfpointorigin}{\pgfpoint{0cm}{-\radius}}}
    \anchor{south east}{\pgfpointadd{\pgfpointorigin}{\pgfpoint{ \radius}{-\radius}}}
    \anchor{south west}{\pgfpointadd{\pgfpointorigin}{\pgfpoint{-\radius}{-\radius}}}
    \anchor{north east}{\pgfpointadd{\pgfpointorigin}{\pgfpoint{ \radius}{ \radius}}}
    \anchor{north west}{\pgfpointadd{\pgfpointorigin}{\pgfpoint{-\radius}{ \radius}}}
    \beforebackgroundpath{
        \pgfsetlinewidth{\pgfkeysvalueof{/tikz/circuits/line width}}
        \pgfpathcircle{\pgfpointorigin}{\radius}
        \pgfusepath{stroke}
        % draw cross
        \pgfpathmoveto{\pgfpointadd{\pgfpointorigin}{\pgfpoint{-0.707107 * \radius}{-0.707107 * \radius}}}
        \pgfpathlineto{\pgfpointadd{\pgfpointorigin}{\pgfpoint{ 0.707107 * \radius}{ 0.707107 * \radius}}}
        \pgfpathmoveto{\pgfpointadd{\pgfpointorigin}{\pgfpoint{-0.707107 * \radius}{ 0.707107 * \radius}}}
        \pgfpathlineto{\pgfpointadd{\pgfpointorigin}{\pgfpoint{ 0.707107 * \radius}{-0.707107 * \radius}}}
        \pgfusepath{stroke}
    }
}
\makeatother

% vim: ft=plaintex

\tikzset{
    circuits/oscillator/radius/.initial = 0.5cm,
    circuits/oscillator/sine width factor/.initial = 0.7,
    circuits/oscillator/sine height factor/.initial = 0.5,
    oscillator/.style = {oscillatorshape}
}

\makeatletter
\pgfdeclareshape{oscillatorshape}
{
    \saveddimen{\radius}{\pgf@x=\pgfkeysvalueof{/tikz/circuits/oscillator/radius}}
    \savedmacro{\sinewidthfactor}{\edef\sinewidthfactor{\pgfkeysvalueof{/tikz/circuits/oscillator/sine width factor}}}
    \savedmacro{\sineheightfactor}{\edef\sineheightfactor{\pgfkeysvalueof{/tikz/circuits/oscillator/sine height factor}}}

    % general anchors
    \anchor{center}{\pgfpointorigin}
    \anchor{west}{\pgfpointadd{\pgfpointorigin}{\pgfpoint{-\radius}{0cm}}}
    \anchor{east}{\pgfpointadd{\pgfpointorigin}{\pgfpoint{\radius}{0cm}}}
    \anchor{north}{\pgfpointadd{\pgfpointorigin}{\pgfpoint{0cm}{ \radius}}}
    \anchor{south}{\pgfpointadd{\pgfpointorigin}{\pgfpoint{0cm}{-\radius}}}
    \anchor{south east}{\pgfpointadd{\pgfpointorigin}{\pgfpoint{ \radius}{-\radius}}}
    \anchor{south west}{\pgfpointadd{\pgfpointorigin}{\pgfpoint{-\radius}{-\radius}}}
    \anchor{north east}{\pgfpointadd{\pgfpointorigin}{\pgfpoint{ \radius}{ \radius}}}
    \anchor{north west}{\pgfpointadd{\pgfpointorigin}{\pgfpoint{-\radius}{ \radius}}}
    \beforebackgroundpath{
        \pgfsetlinewidth{\pgfkeysvalueof{/tikz/circuits/line width}}
        \pgfpathcircle{\pgfpointorigin}{\radius}
        \pgfusepath{stroke}
        % draw sine
        \pgfpathmoveto{\pgfpointadd{\pgfpointorigin}{\pgfpoint{-1 * \sinewidthfactor * \radius}{0.0 * \radius}}}
        \pgfpathsine{\pgfpointadd{\pgfpointorigin}{\pgfpoint{0.5 * \sinewidthfactor * \radius}{\sineheightfactor * \radius}}}
        \pgfpathcosine{\pgfpointadd{\pgfpointorigin}{\pgfpoint{0.5 * \sinewidthfactor * \radius}{-\sineheightfactor * \radius}}}
        \pgfpathsine{\pgfpointadd{\pgfpointorigin}{\pgfpoint{0.5 * \sinewidthfactor * \radius}{-\sineheightfactor * \radius}}}
        \pgfpathcosine{\pgfpointadd{\pgfpointorigin}{\pgfpoint{0.5 * \sinewidthfactor * \radius}{\sineheightfactor * \radius}}}
        \pgfusepath{stroke}
    }
}
\makeatother

% vim: ft=plaintex

\newif\ifantennadrawthrough
\tikzset{
    circuits/antenna/width/.initial = 0.65cm,
    circuits/antenna/head height/.initial = 0.4cm,
    circuits/antenna/foot height/.initial = 0.3cm,
    circuits/antenna/draw through/.is if=antennadrawthrough,
    antenna/.style = {antennashape}
}

\makeatletter
\pgfdeclareshape{antennashape}
{
    \saveddimen{\width}{\pgf@x=\pgfkeysvalueof{/tikz/circuits/antenna/width}}
    \saveddimen{\headheight}{\pgf@x=\pgfkeysvalueof{/tikz/circuits/antenna/head height}}
    \saveddimen{\footheight}{\pgf@x=\pgfkeysvalueof{/tikz/circuits/antenna/foot height}}
    \anchor{center}{\pgfpointorigin}
    \anchor{head center}{
        \pgfpointadd{\pgfpointorigin}{\pgfpoint{0cm}{0.5 * \headheight}}
    }
    \anchor{north}{
        \pgfpointadd{\pgfpointorigin}{\pgfpoint{0cm}{\headheight}}
    }
    \anchor{south}{
        \pgfpointadd{\pgfpointorigin}{\pgfpoint{0cm}{-\footheight}}
    }
    \anchor{south east}{
        \pgfpointadd{\pgfpointorigin}{\pgfpoint{0.5 * \width}{-\footheight}}
    }
    \anchor{north east}{
        \pgfpointadd{\pgfpointorigin}{\pgfpoint{0.5 * \width}{\headheight}}
    }
    \anchor{north west}{
        \pgfpointadd{\pgfpointorigin}{\pgfpoint{-0.5 * \width}{\headheight}}
    }
    \anchor{west}{
        \pgfpointadd{\pgfpointorigin}{\pgfpoint{-0.5 * \width}{0cm}}
    }
    \anchor{east}{
        \pgfpointadd{\pgfpointorigin}{\pgfpoint{0.5 * \width}{0cm}}
    }
    \backgroundpath{
        \pgfsetlinewidth{\pgfkeysvalueof{/tikz/circuits/line width}}
        \pgfpathmoveto{\pgfpointadd{\pgfpointorigin}{\pgfpoint{0cm}{-\footheight}}}
        \ifantennadrawthrough
            \pgfpathmoveto{\pgfpointadd{\pgfpointorigin}{\pgfpoint{0cm}{\headheight}}}
        \else
            \pgfpathlineto{\pgfpointorigin}
        \fi
        \pgfusepath{stroke}
        \pgfpathmoveto{\pgfpointorigin}
        \pgfpathlineto{\pgfpointadd{\pgfpointorigin}{\pgfpoint{0.5 * \width}{\headheight}}}
        \pgfpathlineto{\pgfpointadd{\pgfpointorigin}{\pgfpoint{-0.5 * \width}{\headheight}}}
        \pgfpathclose
        \pgfusepath{stroke}
    }
}
\makeatother

% vim: ft=plaintex nowrap

\newif\ifgyratordrawarrow

\tikzset{
    circuits/gyrator/inner width/.initial = 1.25cm,
    circuits/gyrator/outer width/.initial = 0.75cm,
    circuits/gyrator/height/.initial = 2cm,
    circuits/gyrator/radius/.initial = 1.00cm,
    circuits/gyrator/draw arrow/.is if=gyratordrawarrow,
    gyrator/.style = {gyratorshape}
}

\makeatletter
\pgfdeclareshape{gyratorshape}
{
    \saveddimen{\innerwidth}{\pgf@x=\pgfkeysvalueof{/tikz/circuits/gyrator/inner width}}
    \saveddimen{\outerwidth}{\pgf@x=\pgfkeysvalueof{/tikz/circuits/gyrator/outer width}}
    \saveddimen{\height}{\pgf@x=\pgfkeysvalueof{/tikz/circuits/gyrator/height}}
    \saveddimen{\radius}{\pgf@x=\pgfkeysvalueof{/tikz/circuits/gyrator/radius}}
    \anchor{center}{\pgfpointorigin}
    \anchor{north}{\pgfpointadd{\pgfpointorigin}{\pgfpoint{0cm}{ 0.5 * \height}}}
    \anchor{south}{\pgfpointadd{\pgfpointorigin}{\pgfpoint{0cm}{-0.5 * \height}}}
    \anchor{east}{\pgfpointadd{\pgfpointorigin}{\pgfpoint{ 0.5 * \innerwidth + \outerwidth}{0cm}}}
    \anchor{west}{\pgfpointadd{\pgfpointorigin}{\pgfpoint{-0.5 * \innerwidth - \outerwidth}{0cm}}}
    \anchor{left in upper}{\pgfpointadd{\pgfpointorigin}{\pgfpoint{-0.5 * \innerwidth - \outerwidth}{ 0.5 * \height}}}
    \anchor{left in lower}{\pgfpointadd{\pgfpointorigin}{\pgfpoint{-0.5 * \innerwidth - \outerwidth}{-0.5 * \height}}}
    \anchor{right in upper}{\pgfpointadd{\pgfpointorigin}{\pgfpoint{ 0.5 * \innerwidth + \outerwidth}{ 0.5 * \height}}}
    \anchor{right in lower}{\pgfpointadd{\pgfpointorigin}{\pgfpoint{ 0.5 * \innerwidth + \outerwidth}{-0.5 * \height}}}
    \beforebackgroundpath{
        \pgfsetlinewidth{\pgfkeysvalueof{/tikz/circuits/line width}}
        \pgfpathmoveto{\pgfpointadd{\pgfpointorigin}{\pgfpoint{-0.5 * \innerwidth - 1 * \outerwidth}{ 0.5 * \height}}}
        \pgfpathlineto{\pgfpointadd{\pgfpointorigin}{\pgfpoint{-0.5 * \innerwidth - 0 * \outerwidth}{ 0.5 * \height}}}
        \pgfpathlineto{\pgfpointadd{\pgfpointorigin}{\pgfpoint{-0.5 * \innerwidth - 0 * \outerwidth}{-0.5 * \height}}}
        \pgfpathlineto{\pgfpointadd{\pgfpointorigin}{\pgfpoint{-0.5 * \innerwidth - 1 * \outerwidth}{-0.5 * \height}}}
        \pgfusepath{stroke}
        \pgfpathmoveto{\pgfpointadd{\pgfpointorigin}{\pgfpoint{ 0.5 * \innerwidth + 1 * \outerwidth}{ 0.5 * \height}}}
        \pgfpathlineto{\pgfpointadd{\pgfpointorigin}{\pgfpoint{ 0.5 * \innerwidth + 0 * \outerwidth}{ 0.5 * \height}}}
        \pgfpathlineto{\pgfpointadd{\pgfpointorigin}{\pgfpoint{ 0.5 * \innerwidth + 0 * \outerwidth}{-0.5 * \height}}}
        \pgfpathlineto{\pgfpointadd{\pgfpointorigin}{\pgfpoint{ 0.5 * \innerwidth + 1 * \outerwidth}{-0.5 * \height}}}
        \pgfusepath{stroke}
        % draw circle
        \pgfpathmoveto{\pgfpointadd{\pgfpointorigin}{\pgfpoint{ 0.5 * \innerwidth}{0.5 * \radius}}}
        \pgfpatharc{90}{270}{0.5 * \radius}
        \pgfusepath{stroke}
        \pgfpathmoveto{\pgfpointadd{\pgfpointorigin}{\pgfpoint{-0.5 * \innerwidth}{0.5 * \radius}}}
        \pgfpatharc{90}{-90}{0.5 * \radius}
        \pgfusepath{stroke}
        % draw arrow
        \ifgyratordrawarrow
            \pgfpathmoveto{\pgfpointadd{\pgfpointorigin}{\pgfpoint{-0.4 * \innerwidth}{0.5 * \height}}}
            \pgfpathlineto{\pgfpointadd{\pgfpointorigin}{\pgfpoint{ 0.4 * \innerwidth}{0.5 * \height}}}
            \pgfsetarrows{-{Stealth}}
            \pgfusepath{stroke}
        \fi
    }
}
\makeatother

% vim: ft=plaintex nowrap

\tikzset{
    circuits/inverter/width/.initial = 1.25cm,
    circuits/inverter/height/.initial = 1.25cm,
    circuits/inverter/radius/.initial = 0.1cm,
    inverter/.style = {invertershape}
}

\makeatletter
\pgfdeclareshape{invertershape}
{
    \saveddimen{\width}{\pgf@x=\pgfkeysvalueof{/tikz/circuits/inverter/width}}
    \saveddimen{\height}{\pgf@x=\pgfkeysvalueof{/tikz/circuits/inverter/height}}
    \saveddimen{\radius}{\pgf@x=\pgfkeysvalueof{/tikz/circuits/inverter/radius}}
    \savedanchor{\centerpoint}{\pgfpointorigin}
    \savedanchor{\einput}{%
        \pgfpointadd%
        {\pgfpointorigin}%
        {\pgfpoint%
            {-0.5 * \pgfkeysvalueof{/tikz/circuits/inverter/width}}%
            {0cm}
        }%
    }
    \savedanchor{\output}{%
        \pgfpointadd%
        {\pgfpointorigin}%
        {\pgfpoint%
            {0.5 * \pgfkeysvalueof{/tikz/circuits/inverter/width} + \pgfkeysvalueof{/tikz/circuits/inverter/radius}}%
            {0cm}
        }%
    }
    % electrical terminals (anchors)
    \anchor{in}{\pgfpointadd{\centerpoint}{\pgfpoint{-\width/2}{0cm}}}
    \anchor{out}{\pgfpointadd{\output}{\pgfpoint{\radius}{0cm}}}
    \anchor{+power}{
        \pgfpointlineattime%
        {0.5}%
        {\pgfpointadd{\einput}{\pgfpoint{0cm}{0.5 * \height}}}
        {\pgfpointadd{\output}{\pgfpoint{-\radius}{0cm}}}
    }
    \anchor{-power}{
        \pgfpointlineattime%
        {0.5}%
        {\pgfpointadd{\einput}{\pgfpoint{0cm}{-0.5 * \height}}}
        {\pgfpointadd{\output}{\pgfpoint{-\radius}{0cm}}}
    }
    % regular anchors
    \anchor{center}{\centerpoint}
    \anchor{north}{
        \pgfpointadd{\pgfpointorigin}{\pgfpoint{0cm}{0.5 * \height}}
    }
    \anchor{south}{
        \pgfpointadd{\pgfpointorigin}{\pgfpoint{0cm}{-0.5 * \height}}
    }
    \anchor{east}{
        \pgfpointadd{\pgfpointorigin}{\pgfpoint{\width/2 + 2 * \radius}{0cm}}
    }
    \anchor{west}{
        \pgfpointadd{\pgfpointorigin}{\pgfpoint{-\width/2}{0cm}}
    }
    \anchor{north west}{
        \pgfpointadd{\einput}{\pgfpoint{0cm}{0.5 * \height}}
    }
    \anchor{south west}{
        \pgfpointadd{\pgfpointorigin}{\pgfpoint{-0.5 * \width}{-0.5 * \height}}
    }
    \anchor{north east}{
        \pgfpointadd{\einput}{\pgfpoint{\width + 2 * \radius}{0.5 * \height}}
    }
    \anchor{south east}{
        \pgfpointadd{\pgfpointorigin}{\pgfpoint{0.5 * \width + 2 * \radius}{-0.5 * \height}}
    }
    \backgroundpath{
        \pgfsetlinewidth{\pgfkeysvalueof{/tikz/circuits/line width}}
        \pgfsetmiterjoin
        \pgfsetmiterlimit{2}
        \pgfpathmoveto{\pgfpointadd{\pgfpointorigin}{\pgfpoint{-0.5 * \width}{-0.5 * \height}}}
        \pgfpathlineto{\pgfpointadd{\output}{\pgfpoint{-\radius}{0cm}}}
        \pgfpathlineto{\pgfpointadd{\pgfpointorigin}{\pgfpoint{-0.5 * \width}{ 0.5 * \height}}}
        \pgfpathclose
        \pgfusepath{stroke}
        \pgfpathcircle{
            \pgfpointadd%
            {\pgfpointorigin}%
            {\pgfpoint%
                {0.5 * \pgfkeysvalueof{/tikz/circuits/inverter/width} + \pgfkeysvalueof{/tikz/circuits/inverter/radius}}%
                {0cm}
            }%
        }{\radius}
        \pgfusepath{stroke}
    }
}
\makeatother

% vim: ft=plaintex nowrap


% vim: ft=plaintex nowrap
